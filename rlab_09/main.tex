%!TEX TS-program = xelatex

% Шаблон документа LaTeX создан в 2018 году
% Алексеем Подчезерцевым
% В качестве исходных использованы шаблоны
% 	Данилом Фёдоровых (danil@fedorovykh.ru) 
%		https://www.writelatex.com/coursera/latex/5.2.2
%	LaTeX-шаблон для русской кандидатской диссертации и её автореферата.
%		https://github.com/AndreyAkinshin/Russian-Phd-LaTeX-Dissertation-Template

\documentclass[a4paper,14pt]{article}

\input{data/preambular.tex}
\begin{document} % конец преамбулы, начало документа
\begin{titlepage}
	\begin{center}
		ФЕДЕРАЛЬНОЕ  ГОСУДАРСТВЕННОЕ АВТОНОМНОЕ \\
		ОБРАЗОВАТЕЛЬНОЕ УЧРЕЖДЕНИЕ ВЫСШЕГО ОБРАЗОВАНИЯ\\
		«НАЦИОНАЛЬНЫЙ ИССЛЕДОВАТЕЛЬСКИЙ УНИВЕРСИТЕТ\\
		«ВЫСШАЯ ШКОЛА ЭКОНОМИКИ»
	\end{center}
	
	\begin{center}
		\textbf{Московский институт электроники и математики}
		
		\textbf{Им. А.Н.Тихонова НИУ ВШЭ}
		
		\textbf{Департамент компьютерной инженерии}
	\end{center}
	\vspace{1ex}	
	\begin{center}
		Подчезерцев Алексей Евгеньевич, группа БИВ174
		
		Солодянкин Андрей Александрович, группа БИВ174
	\end{center}	
	\vspace{1ex}
	\begin{center}
		\textbf{ОТЧЕТ\\
		ПО ЛАБОРАТОРНОЙ РАБОТЕ №1
	}
	\end{center}	
	\vspace{2ex}
	\begin{center}
		по дисциплине «Проектирование систем на кристалле»
	\end{center}

	\vfill
	\begin{center}
		Москва \the\year \, г.
	\end{center}
\end{titlepage}
\tableofcontents
\pagebreak

\section{Задание}

Опишите идею прикладного программно-аппаратного проекта с использованием ПЛИС или одноплатного компьютера.

\section{Идея проекта}

Трекинг позиции окончания пишущего устройства на бумаге для непрерывной оцифровки рукописи человека.
В условиях удаленного обучения часть студентов испытывает трудности с выступлением у виртуальной <<доски>>, так как они не могут с той же скоростью и качеством писать что-либо с помощью устройства позиционирования курсора операционной системы. Данная проблема тратит время преподавателя на семинарах и других типах занятий, студентов, которые наблюдают за занятием, непосредственно выступающего учащегося.
Данное решение позволит экономить время всех участников и не тратить большие финансовые вложения в графические планшеты, которые могут и не пригодиться в будущем.

\section{Заказчик проекта}

Потенциальные заказчики -- Московский институт электроники и математики им. А.Н. Тихонова, коммерческая компания, стартап.

\section{Потребитель проекта}

Студенты; люди, не имеющие возможности приобрести графический планшет ввиду его большой стоимости и крайнего узкого спектра применения.

\section{Польза проекта}

Проект позволяет в реальном времени отслеживает движение пишущего устройства (ручки, карандаша и других канцелярский принадлежностей) по поверхности бумаги и оцифровывает полученные кривые в пригодный для отображения на персональном компьютере.
В отличии от графического планшета стоимость устройства должна быть заметно ниже.
В отличии от устройств позиционирования курсора для персонального компьютера, которые не позволяют быстро и качественно отображать формулы и другую графическую информацию, данное устройство позволит выводить информацию, которая создается естественным путем -- написанием от руки.

\section{Схема проекта}

Схема проекта представлена на рис.~\ref{fig:schema}.
Камера осуществляет захват изображения и передает его на одноплатный компьютер или ПЛИС.
Последнее устройство осуществляет обработку видеоинформации, занимается поиском нужных объектов на видеоизображении и строит распознанные кривые.
Полученные данные передаются на персональный компьютер или мобильное устройство пользователя посредством проводного или беспроводного протокола передачи данных в виде векторного или растрового изображения или в виде команд для устройства позиционирования указателя операционной системы пользователя -- курсора.

\begin{figure}[H]
	\centering
	\includegraphics[width=\linewidth]{image/schema}
	\caption{Схема проекта}
	\label{fig:schema}
\end{figure}

\section{Составляющие проекта}

Первая составляющая проекта, которая выполняет захват изображения -- камера.
Следующая составляющая проекта -- ПЛИС или одноплатный компьютер.
Данное устройство выполняет обработку видеоинформации, осуществляет поиск необходимых объектов на видеоизображении и строит итоговые кривые.
Передача данных между данными устройствами осуществляется по протоколу камеры (USB, Wi-Fi, etc.).

Передача данных от устройства выполняется путем использования протоколов проводной передачи данных (USB) или беспроводной передачи данных: Wi-Fi, Bluetooth.
Конечное решение принимает команда разработки в зависимости от сложности, стоимости и удобства использования того или иного протокола.

\section{Шаги проекта}

Для успешной реализации проекта необходимо пройти следующие этапы:

\begin{itemize}
	\item Разработка программного прототипа.

	На данном этапе необходимо подтвердить работоспособность идеи отслеживания позиции пишущего устройства, оценить затраты на вычислительные ресурсы и качество получаемого решения.
	Одновременно необходимо выбрать оборудование для захвата видеоряда и собрать видеозаписи, на которых будет проверяться качество полученного алгоритма.
	
	\item Выбор одноплатного компьютера или ПЛИС и перенос кода.
	
	На данном этапе необходимо выбрать устройство, которое способно обрабатывать видеоряд в реальном времени в соответствии с затратами алгоритма и его реализацией в коде на данном устройстве.
	
	\item Создание финального прототипа.
	
	На данном этапе команда проекта создает дизайн внешнего вида устройства и оформляет как законченный продукт. 
	Также проходит тестирование устройства на соответствие поставленным задачам и получаемым результатам.
	
	\item Демонстрация проекта, серийное производство.
	
	Команда проекта демонстрирует полученные результаты, занимается поиском возможных инвесторов, готовится в массовому производству.
	
\end{itemize}

\section{Аппаратная платформа проекта}

В качестве аппаратной платформы используется De10-Nano или другой ПЛИС, или Raspberry Pi 4 или другой одноплатный компьютер или устройство другой архитектуры, в зависимости от требований проекта, например мобильные устройства или вычислительные системы архитектуры x86.
В качестве камеры используется устройство с разрешением не менее 1280x720 и совместимым с выбранной аппаратной платформой.

\section{Затраты проекта}

Список затрат проекта (без учета оплаты труда команды разработчиков) представлен на таблице~\ref{tab:my-table}.
На первом этапе основной источник затрат -- камера, а также запись датасетов для обучения алгоритмов.
На следующем этапе потребуются финансы для тестирования алгоритмов на выбранном железе.
При создании финального прототипа потребуется создание корпуса и демонстрационных материалов; оплаты труда тестировщиков полученного устройства.
На финальном этапе трудно оценить затраты, но часть из них может пойти на доработку системы, создания большего числа прототипов, рекламу и другие цели.

\begin{table}[H]
	\caption{Список затрат на каждый этап проекта}
	\label{tab:my-table}
	\begin{tabular}{|c|c|c|}
		\hline
		Этап                 & Временные затраты, мес & Финансовые затраты, 1000₽ \\ \hline
		Разработка прототипа & 2-3                    & 10                        \\ \hline
		Выбор платформы      & 1-2                    & 30                        \\ \hline
		Финальный прототип   & 0,5-1                  & 20                        \\ \hline
		Демо                 & 3+                     & 50+                       \\ \hline
	\end{tabular}
\end{table}

\end{document} % конец документа