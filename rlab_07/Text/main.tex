%!TEX TS-program = xelatex

% Шаблон документа LaTeX создан в 2018 году
% Алексеем Подчезерцевым
% В качестве исходных использованы шаблоны
% 	Данилом Фёдоровых (danil@fedorovykh.ru) 
%		https://www.writelatex.com/coursera/latex/5.2.2
%	LaTeX-шаблон для русской кандидатской диссертации и её автореферата.
%		https://github.com/AndreyAkinshin/Russian-Phd-LaTeX-Dissertation-Template

\documentclass[a4paper,14pt]{article}

\input{data/preambular.tex}
\begin{document} % конец преамбулы, начало документа
\begin{titlepage}
	\begin{center}
		ФЕДЕРАЛЬНОЕ  ГОСУДАРСТВЕННОЕ АВТОНОМНОЕ \\
		ОБРАЗОВАТЕЛЬНОЕ УЧРЕЖДЕНИЕ ВЫСШЕГО ОБРАЗОВАНИЯ\\
		«НАЦИОНАЛЬНЫЙ ИССЛЕДОВАТЕЛЬСКИЙ УНИВЕРСИТЕТ\\
		«ВЫСШАЯ ШКОЛА ЭКОНОМИКИ»
	\end{center}
	
	\begin{center}
		\textbf{Московский институт электроники и математики}
		
		\textbf{Им. А.Н.Тихонова НИУ ВШЭ}
		
		\textbf{Департамент компьютерной инженерии}
	\end{center}
	\vspace{1ex}	
	\begin{center}
		Подчезерцев Алексей Евгеньевич, группа БИВ174
		
		Солодянкин Андрей Александрович, группа БИВ174
	\end{center}	
	\vspace{1ex}
	\begin{center}
		\textbf{ОТЧЕТ\\
		ПО ЛАБОРАТОРНОЙ РАБОТЕ №1
	}
	\end{center}	
	\vspace{2ex}
	\begin{center}
		по дисциплине «Проектирование систем на кристалле»
	\end{center}

	\vfill
	\begin{center}
		Москва \the\year \, г.
	\end{center}
\end{titlepage}
\tableofcontents
\pagebreak

\section{Задание}

Создать сумматор-накопитель, входные данные A и B по 8 бит, поступают из RAM. 
По разрешающему сигналу данные складываются и добавляются к предыдущему результату.
Предусмотреть сброс данных в накопителе.

\section{Выполнение работы}

Создадим проект, в качестве device выберем 10M50DCF484C7G семейства MAX 10.

Добавим в проект IP блок двупортовой памяти (рис.~\ref{fig:ram_1}) и настроим его (рис.~\ref{fig:ram_2}).
Также добавим блок параллельного сумматора (рис.~\ref*{fig:add}).

\begin{figure}[H]
	\centering
	\includegraphics[width=\linewidth]{image/ram_1}
	\caption{Создание блока RAM}
	\label{fig:ram_1}
\end{figure}

\begin{figure}[H]
	\centering
	\includegraphics[width=\linewidth]{image/ram_2}
	\caption{Настройка блока RAM}
	\label{fig:ram_2}
\end{figure}

\begin{figure}[H]
	\centering
	\includegraphics[width=\linewidth]{image/add}
	\caption{Создание блока сумматора}
	\label{fig:add}
\end{figure}

Создадим схему проекта (рис~\ref{fig:schema}).
Входные данные записываются в двупортовую память, оттуда поступают на вход сумматора, результат записывается в триггер и дублируется на вход сумматора.
Предусмотрен сигнал сброса накопленного результата.

\begin{figure}[H]
	\centering
	\includegraphics[width=\linewidth]{image/schema}
	\caption{Схема проекта}
	\label{fig:schema}
\end{figure}

Выполним компиляцию проекта.
В отчете компиляции (рис.~\ref{fig:report}) видно потребление логических и других элементов данным проектом.
Отчет о потреблении элементов на блок представлен на рис.~\ref{fig:elements}

\begin{figure}[H]
	\centering
	\includegraphics[width=0.6\linewidth]{image/report}
	\caption{Результат компиляции}
	\label{fig:report}
\end{figure}

\begin{figure}[H]
	\centering
	\includegraphics[width=0.6\linewidth]{image/elements}
	\caption{Потребление элементов}
	\label{fig:elements}
\end{figure}

Выполним тестирование схемы с помощью waveform.
На вход сумматора подаются попеременно пары чисел 1, 1 и 1, 3(рис.~\ref{fig:wave}).
Как можно заметить, результат суммирования увеличивается на 2 или 4 в зависимости от входных значений.
При подаче сигнала сброса результат вычислений выставляется в 0.

\begin{figure}[H]
	\centering
	\includegraphics[width=\linewidth]{image/wave}
	\caption{Тестирование схемы}
	\label{fig:wave}
\end{figure}

Построим RTL представление проекта (рис.~\ref{fig:rtl}).

\begin{figure}[H]
	\centering
	\includegraphics[width=\linewidth]{image/rtl}
	\caption{RTL представление проекта}
	\label{fig:rtl}
\end{figure}

Из RTL представления проекта перейдем в TMV представление памяти (рис.~\ref{fig:tmv}).

\begin{figure}[H]
	\centering
	\includegraphics[width=\linewidth]{image/tmv}
	\caption{TMV представление памяти}
	\label{fig:tmv}
\end{figure}

Далее перейдем в chip planner и рассмотрим подробно входы (рис.~\ref{fig:mem_in}) и выходы (рис.~\ref{fig:mem_out}) памяти.

\begin{figure}[H]
	\centering
	\includegraphics[width=0.3\linewidth]{image/mem_in}
	\caption{Входы в память}
	\label{fig:mem_in}
\end{figure}

\begin{figure}[H]
	\centering
	\includegraphics[width=0.3\linewidth]{image/mem_out}
	\caption{Выходы памяти}
	\label{fig:mem_out}
\end{figure}

Расположим пины по банкам, используемые в проекте пины подсвечены зеленым (рис.~\ref{fig:pins}).

\begin{figure}[H]
	\centering
	\includegraphics[width=0.8\linewidth]{image/pins}
	\caption{Назначение пинов}
	\label{fig:pins}
\end{figure}

Создадим новую ревизию проекта, настроим ее на оптимизацию энергопотребления (рис.~\ref{fig:settings}).
Выполним повторную сборку проекта, посмотрим на различия ревизий (рис.~\ref{fig:diff}).

\begin{figure}[H]
	\centering
	\includegraphics[width=0.8\linewidth]{image/settings}
	\caption{Настройка новой ревизии}
	\label{fig:settings}
\end{figure}

\begin{figure}[H]
	\centering
	\includegraphics[width=0.8\linewidth]{image/diff}
	\caption{Сравнение ревизий}
	\label{fig:diff}
\end{figure}

\section{Выводы по работе}

В ходе работы был получен блок накопительного сложения чисел с возможностью сброса результата.
Устройство были протестированы с помощью Quartus waveform, что подтвердило правильность работоспособности устройства.
Так же было выполнено построение RTL и TMV диаграмм для устройства, построены зависимости в chip planer, настроены пины.

\newpage 
\renewcommand{\refname}{{\normalsize Список использованных источников}} 
\centering 
\begin{thebibliography}{9} 
	\addcontentsline{toc}{section}{\refname} 
	\bibitem{Verilog} Thomas D., Moorby P. The Verilog Hardware Description Language. – Springer Science \& Business Media, 2008.
	\bibitem{Quartus} Антонов А., Филиппов А., Золотухо Р. Средства системной отладки САПР Quartus II //Компоненты и технологии. – 2008. – №. 89.
\end{thebibliography}

\end{document} % конец документа