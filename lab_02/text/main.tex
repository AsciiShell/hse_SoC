%!TEX TS-program = xelatex

% Шаблон документа LaTeX создан в 2018 году
% Алексеем Подчезерцевым
% В качестве исходных использованы шаблоны
% 	Данилом Фёдоровых (danil@fedorovykh.ru) 
%		https://www.writelatex.com/coursera/latex/5.2.2
%	LaTeX-шаблон для русской кандидатской диссертации и её автореферата.
%		https://github.com/AndreyAkinshin/Russian-Phd-LaTeX-Dissertation-Template

\documentclass[a4paper,14pt]{article}

\input{data/preambular.tex}
\begin{document} % конец преамбулы, начало документа
\begin{titlepage}
	\begin{center}
		ФЕДЕРАЛЬНОЕ  ГОСУДАРСТВЕННОЕ АВТОНОМНОЕ \\
		ОБРАЗОВАТЕЛЬНОЕ УЧРЕЖДЕНИЕ ВЫСШЕГО ОБРАЗОВАНИЯ\\
		«НАЦИОНАЛЬНЫЙ ИССЛЕДОВАТЕЛЬСКИЙ УНИВЕРСИТЕТ\\
		«ВЫСШАЯ ШКОЛА ЭКОНОМИКИ»
	\end{center}
	
	\begin{center}
		\textbf{Московский институт электроники и математики}
		
		\textbf{Им. А.Н.Тихонова НИУ ВШЭ}
		
		\textbf{Департамент компьютерной инженерии}
	\end{center}
	\vspace{1ex}	
	\begin{center}
		Подчезерцев Алексей Евгеньевич, группа БИВ174
		
		Солодянкин Андрей Александрович, группа БИВ174
	\end{center}	
	\vspace{1ex}
	\begin{center}
		\textbf{ОТЧЕТ\\
		ПО ЛАБОРАТОРНОЙ РАБОТЕ №1
	}
	\end{center}	
	\vspace{2ex}
	\begin{center}
		по дисциплине «Проектирование систем на кристалле»
	\end{center}

	\vfill
	\begin{center}
		Москва \the\year \, г.
	\end{center}
\end{titlepage}
\tableofcontents
\pagebreak
\section{Задание}

\begin{enumerate}
	\item Используя опыт, полученный при создании двухступенчатого RS-триггера, самостоятельно
	разработайте на языке Verilog модуль, описывающий структуру JК-триггера, приведенную на
	Рисунке 2.17.
	
	
	\item Используя созданные ранее D-триrгер и JК-триггер, опишите Verilog модули Т-триггера в
	соответствии с Рисунком 2.18 и самостоятельно постройте таблицу переходов данного
	триггера.
	
	\item Найти в сети Интернет документ с рекомендациями по программированию на языке HDL для
	ПЛИС производства компании Intel FPGA.
	
	\item Отредактируйте код D-защелки (Листинг 2.16) так, чтобы входное значение сохранялось при
	низком уровне тактового сигнала.
	Скомпилируйте код и сравните его RТL-представление с Рисунком 2.23.
	
	\item Измените D-триггер, код которого приведен в Листинге 2.17 так, чтобы он работал под
	управлением заднего фронта тактового сигнала. 
	Для этого необходимо заменить ключевое слово posedge на negedge. 
	Скомпилируйте проект и сравните его RТL-представление с	предыдущей реализацией на Рисунке 2.25.
	
	\item Используя опыт, полученный при реализации D-триггера, самостоятельно отредактируйте
	код Листинга 2.19. 
	Проведите его моделирование, имитируя все возможные комбинации входных сигналов, приведенные в таблице переходов на Рисунке 2.16.
	Сравните получаемые	результаты с этой таблицей.
	
	\item Аналогичным образом отредактируйте код Листинга 2.18, добавив еще одну кнопку.
	Проверьте работоспособность JК-триггера, наблюдая за состоянием светодиода,	подключенного к выходу. 
	Сравните затраты ресурсов, необходимых для создания D-триггера и JК-триггера в RТL Viewer.
	
	\item Неинвертированный синхронный сигнал сброса часто называют сигналом очистки.
	Измените пример в Листинге 2.20 для реализации D-триггера с сигналом очистки.
	Сравните его RТL-представление с Рисунком 2.29.
	
	\item Добавьте в D-триггер асинхронный сброс с сигналом очистки, реализованным на предыдущем шаге.
	Модуль должен иметь положительный синхронный сигнал очистки и в то же время асинхронный вход n\_rst.
	Сравните е го аппаратную реализацию (RТL-представление) с Рисунком 2.31.
	
	\item Измените активируемый триггер в Листинге 2.24.
	Необходимо добавить синхронный положительный сигнал очистки и асинхронный сигнал n\_rst.
	Сравните аппаратную реализацию (RTL View) с Рисунком 2.35.
	
	\item Измените реализацию триггера, приведенную в Листинге 2.28. 
	Добавьте положительный сигнал синхронной очистки, асинхронный сброс n\_rst и сигнал разрешения работы en.
	Сравните аппаратную реализацию (RTL View) с Рисунком 2.37.
\end{enumerate}

\section{Выполнение работы}

\subsection{JK-триггер}

\VerbatimInput{../jk_latch.v}

Временная диаграмма для JK-триггера изображена на рис.\ref{fig:jk_wave}.

\begin{figure}[H]
	\centering
	\includegraphics[width=0.95\linewidth]{imgs/jk_wave}
	\caption{Временная диаграмма для JK-триггера}
	\label{fig:jk_wave}
\end{figure}


\subsection{T-триггер}

\VerbatimInput{../t_latch.v}

\begin{table}[H]
	\begin{center}
		\begin{flushleft}
			\tablecaption{Таблица переходов для T-триггера}
		\end{flushleft}
		\label{tab:t_flip_flop}
		\begin{tabular}{|c|c|}
			\hline
			$T$ & $Q$                   \\ \hline
			0   & $Q_{prev}$            \\ \hline
			1   & $\overline{Q_{prev}}$ \\ \hline
		\end{tabular}
	\end{center}
\end{table}

Результаты работы Т-триггера на рис.\ref{fig:t_wave2}, \ref{fig:t_wave_sync}.
\begin{figure}[H]
	\centering
	\includegraphics[width=0.7\linewidth]{imgs/t_wave_sync}
	\caption{Т-триггер в обычном режиме}
	\label{fig:t_wave_sync}
\end{figure}

\begin{figure}[H]
	\centering
	\includegraphics[width=0.7\linewidth]{imgs/t_wave2}
	\caption{T-триггер в режиме понижения частоты}
	\label{fig:t_wave2}
\end{figure}

\subsection{Документация HDL}

Документация найдена[\cite{citekey}].

\subsection{D-защелка}

\VerbatimInput{../d_latch.v}

На рис.\ref{fig:4_RTL_norm}, \ref{fig:4_RTL_inv} можно сравнить RTL диаграммы для данных схем.

\begin{figure}[H]
	\centering
	\includegraphics[width=0.7\linewidth]{imgs/4_RTL_norm}
	\caption{D-защелка с низким уровнем сигнала}
	\label{fig:4_RTL_norm}
\end{figure}

\begin{figure}[H]
	\centering
	\includegraphics[width=0.7\linewidth]{imgs/4_RTL_inv}
	\caption{D-защелка с высоким уровнем сигнала}
	\label{fig:4_RTL_inv}
\end{figure}

\subsection{D-триггер}

\VerbatimInput{../d_flip_flop.v}

На рис.\ref{fig:5_RTL_posedge_sample}, \ref{fig:5_RTL_negedge} можно сравнить RTL диаграммы для данных схем.

\begin{figure}[H]
	\centering
	\includegraphics[width=0.7\linewidth]{imgs/5_RTL_posedge_sample}
	\caption{D-триггер с передним фронтом}
	\label{fig:5_RTL_posedge_sample}
\end{figure}

\begin{figure}[H]
	\centering
	\includegraphics[width=0.7\linewidth]{imgs/5_RTL_negedge}
	\caption{D-триггер с задним фронтом}
	\label{fig:5_RTL_negedge}
\end{figure}


\subsection{JK-триггер}

\begin{table}[H]
	\begin{center}
	\begin{flushleft}
		\tablecaption{Таблица переходов для JK-триггера}
	\end{flushleft}
		\label{tab:jk_flip_flop}
		\begin{tabular}{|c|c|c|}
			\hline
			J & K & Q                     \\ \hline
			0 & 0 & $Q_{prev}$            \\ \hline
			1 & 0 & 1                     \\ \hline
			0 & 1 & 0                     \\ \hline
			1 & 1 & $\overline{Q_{prev}}$ \\ \hline
		\end{tabular}
	\end{center}
\end{table}

\VerbatimInput{../jk_flip_flop.v}

На рис.\ref{fig:6_RTL_jk}, \ref{fig:6_wave_jk} отображена RTL диаграмма и результат тестирования триггера.

\begin{figure}[H]
	\centering
	\includegraphics[width=0.9\linewidth]{imgs/6_RTL_jk}
	\caption{JK-триггер}
	\label{fig:6_RTL_jk}
\end{figure}

\begin{figure}[H]
	\centering
	\includegraphics[width=0.9\linewidth]{imgs/6_wave_jk}
	\caption{Тестирование триггера}
	\label{fig:6_wave_jk}
\end{figure}

\VerbatimInput{../de10_lite_jk_flip_flop.v}

\subsection{JK-триггер с синхронным сигналом}

На рис.\ref{fig:7_RTL_sample}, \ref{fig:7_RTL_neinv} отображена

\begin{figure}[H]
	\centering
	\includegraphics[width=0.7\linewidth]{imgs/7_RTL_sample}
	\caption{Обычный сигнал}
	\label{fig:7_RTL_sample}
\end{figure}

\begin{figure}[H]
	\centering
	\includegraphics[width=0.7\linewidth]{imgs/7_RTL_neinv}
	\caption{Инвертированный сигнал}
	\label{fig:7_RTL_neinv}
\end{figure}

\subsection{D-триггер с очисткой}

На рис.\ref{fig:8_RTL_sample}, \ref{fig:8_RTL_neinv} отображена

\begin{figure}[H]
	\centering
	\includegraphics[width=0.7\linewidth]{imgs/8_RTL_sample}
	\caption{Обычный сигнал}
	\label{fig:8_RTL_sample}
\end{figure}

\begin{figure}[H]
	\centering
	\includegraphics[width=0.7\linewidth]{imgs/8_RTL_neinv}
	\caption{Инвертированный сигнал}
	\label{fig:8_RTL_neinv}
\end{figure}

\subsection{D-триггер со сбросом}

% Сравнение RTL

\subsection{Какая то ересь в №10}

% Сравнение RTL

\subsection{Какая то ересь в №11}

% Сравнение RTL

\section{Контрольные вопросы}

\begin{enumerate}
	\item b) D-тpиrrep;
	\item AZ, BY, CX;
	\item Задержка распространения: $t_{pd}$ = максимальная задержка от входа к выходу;
	\item Синхронные триггеры позволяют изменять значение своего состояния только при подаче специального сигнала.
	Асинхронные могут изменять свое состояние в любой момент времени;
	\item Запрещенная комбинация -- комбинация входных сигналов, подача которых на вход триггера приведет его в неопределенное состояние;
	При подаче запрещенной комбинации на обоих выходах триггера будет низкий логический сигнал, при чем триггер установится в определенное состояние при подаче другого сигнала;
	\item D-триггер синхронный, в отличии от D-защелки, т.е. обновляет свое значение только при подаче сигнала;
	\item В схеме slave-master используется две D-защелки, когда схема на бистабильных ячейках более сложная и состоит из нескольких логических элементов с обратными связями;
	\item Время установки -- это время перед приходом перепада сигнала CLK, в течение которого сигнал D должен быть стабилен;
	\item Время удержания -- это время, в течение которого сигнал D должен быть стабилен, после прихода перепада сигнала CLK;
	\item На JК-триггер можно реализовать D-триггер и T-триггер;
	\item Разработайте схему Т-триггера на основе D-триггера (рис. \ref{fig:self_made_t_on_d});
	\begin{figure}[H]
		\centering
		\includegraphics[width=0.6\linewidth]{imgs/self_made_t_on_d}
		\caption{Т-триггера на основе D-триггера}
		\label{fig:self_made_t_on_d}
	\end{figure}
	\item Операторы непрерывного присваивания на языке Verilog соединяет сигналы внутри схемы;
	\item Блокирующие присваивания выполняются последовательно, один за другим.
	Неблокирующие присваивания выполняются одновременно;
	\item Синхронный сброс может произойти только по фронту синхронизирующего сигнала, в то время как асинхронный сброс может произойти в любой момент времени;

	Пример синхронизирующего сброса:
	\begin{verbatim}
	module dff_sync_rst_n
	(	
		input clk,
		input rst_n,
		input d,
		output reg q
	) ;
	always @ (posedge clk)
		if (!rst_n)
			q <= О;
		else
			q <= d;
	endmodule
	\end{verbatim}
	
	\item Ключеве слова parameter в языке Verilog позволяет переиспользовать исходный код при разработке модулей разной разрядности. 
	С помощью данного параметра можно настраивать размерность модулей;
	\item $ P = \alpha C U^2 f$;
	
	$\alpha$ -- среднее число переключений в течении тактового периода;
	
	$C$ -- емкость нагрузки;
	
	$U$ -- напряжение питания;
	
	$f$ -- частота тактового сигнала;
	
	\item Ячейка блокировки -- ячейка, которая блокирует изменения состояния неиспользуемых элементов. 
	Данная ячейка применяется для снижения энергопотребления логического элемента;
	\item Блокировка тактового сигнала может привести к появлению шумов в линии тактового сигнала и ложного срабатывания управляемого элемента.
\end{enumerate}

\section{Выводы по работе}

В ходе работы получен опыт проектирования схем в программе Quartus с помощью языка Verilog.
Полученное устройство было протестировано с помощью бенчтестов в программе Quartus Simulation Waveform editor.
В процессе работы были смоделированы различные триггеры с синхронной и асинхронным управлением, сигналами сброса.
В процессе был получен опыт работы с платой DE10-Lite, на которой проверялась работоспособность полученного устройства.

\newpage 
\renewcommand{\refname}{{\normalsize Список использованных источников}} 
\centering 
\begin{thebibliography}{9} 
	\addcontentsline{toc}{section}{\refname} 
	\bibitem{Verilog} Thomas D., Moorby P. The Verilog Hardware Description Language. – Springer Science \& Business Media, 2008.
	\bibitem{citekey} Khor W. Y. et al. Evaluation of FPGA Based QSPI Flash Access Using Partial Reconfiguration //2019 7th International Conference on Smart Computing \& Communications (ICSCC). – IEEE, 2019. – С. 1-5
\end{thebibliography}

\end{document} % конец документа






