%!TEX TS-program = xelatex

% Шаблон документа LaTeX создан в 2018 году
% Алексеем Подчезерцевым
% В качестве исходных использованы шаблоны
% 	Данилом Фёдоровых (danil@fedorovykh.ru) 
%		https://www.writelatex.com/coursera/latex/5.2.2
%	LaTeX-шаблон для русской кандидатской диссертации и её автореферата.
%		https://github.com/AndreyAkinshin/Russian-Phd-LaTeX-Dissertation-Template

\documentclass[a4paper,14pt]{article}

\input{data/preambular.tex}
\begin{document} % конец преамбулы, начало документа
\begin{titlepage}
	\begin{center}
		ФЕДЕРАЛЬНОЕ  ГОСУДАРСТВЕННОЕ АВТОНОМНОЕ \\
		ОБРАЗОВАТЕЛЬНОЕ УЧРЕЖДЕНИЕ ВЫСШЕГО ОБРАЗОВАНИЯ\\
		«НАЦИОНАЛЬНЫЙ ИССЛЕДОВАТЕЛЬСКИЙ УНИВЕРСИТЕТ\\
		«ВЫСШАЯ ШКОЛА ЭКОНОМИКИ»
	\end{center}
	
	\begin{center}
		\textbf{Московский институт электроники и математики}
		
		\textbf{Им. А.Н.Тихонова НИУ ВШЭ}
		
		\textbf{Департамент компьютерной инженерии}
	\end{center}
	\vspace{1ex}	
	\begin{center}
		Подчезерцев Алексей Евгеньевич, группа БИВ174
		
		Солодянкин Андрей Александрович, группа БИВ174
	\end{center}	
	\vspace{1ex}
	\begin{center}
		\textbf{ОТЧЕТ\\
		ПО ЛАБОРАТОРНОЙ РАБОТЕ №1
	}
	\end{center}	
	\vspace{2ex}
	\begin{center}
		по дисциплине «Проектирование систем на кристалле»
	\end{center}

	\vfill
	\begin{center}
		Москва \the\year \, г.
	\end{center}
\end{titlepage}
\tableofcontents
\pagebreak
\section{Задание}

\begin{enumerate}
	\item Используя опыт, полученный при создании двухступенчатого RS-триггера, самостоятельно
	разработайте на языке Verilog модуль, описывающий структуру JК-триггера, приведенную на
	Рисунке 2.17.
	
	
	\item Используя созданные ранее D-триrгер и JК-триггер, опишите Verilog модули Т-триггера в
	соответствии с Рисунком 2.18 и самостоятельно постройте таблицу переходов данного
	триггера.
	
	\item Найти в сети Интернет документ с рекомендациями по программированию на языке HDL для
	ПЛИС производства компании Intel FPGA.
	
	\item Отредактируйте код D-защелки (Листинг 2.16) так, чтобы входное значение сохранялось при
	низком уровне тактового сигнала.
	Скомпилируйте код и сравните его RТL-представление с Рисунком 2.23.
	
	\item Измените D-триггер, код которого приведен в Листинге 2.17 так, чтобы он работал под
	управлением заднего фронта тактового сигнала. 
	Для этого необходимо заменить ключевое слово posedge на negedge. 
	Скомпилируйте проект и сравните его RТL-представление с	предыдущей реализацией на Рисунке 2.25.
	
	\item Используя опыт, полученный при реализации D-триггера, самостоятельно отредактируйте
	код Листинга 2.19. 
	Проведите его моделирование, имитируя все возможные комбинации входных сигналов, приведенные в таблице переходов на Рисунке 2.16.
	Сравните получаемые	результаты с этой таблицей.
	
	\item Аналогичным образом отредактируйте код Листинга 2.18, добавив еще одну кнопку.
	Проверьте работоспособность JК-триггера, наблюдая за состоянием светодиода,	подключенного к выходу. 
	Сравните затраты ресурсов, необходимых для создания D-триггера и JК-триггера в RТL Viewer.
	
	\item Неинвертированный синхронный сигнал сброса часто называют сигналом очистки.
	Измените пример в Листинге 2.20 для реализации D-триггера с сигналом очистки.
	Сравните его RТL-представление с Рисунком 2.29.
	
	\item Добавьте в D-триггер асинхронный сброс с сигналом очистки, реализованным на предыдущем шаге.
	Модуль должен иметь положительный синхронный сигнал очистки и в то же время асинхронный вход n\_rst.
	Сравните е го аппаратную реализацию (RТL-представление) с Рисунком 2.31.
	
	\item Измените активируемый триггер в Листинге 2.24.
	Необходимо добавить синхронный положительный сигнал очистки и асинхронный сигнал n\_rst.
	Сравните аппаратную реализацию (RTL View) с Рисунком 2.35.
	
	\item Измените реализацию триггера, приведенную в Листинге 2.28. 
	Добавьте положительный сигнал синхронной очистки, асинхронный сброс n\_rst и сигнал разрешения работы en.
	Сравните аппаратную реализацию (RTL View) с Рисунком 2.37.
\end{enumerate}

\section{Выполнение работы}

\subsection{JK-триггер}

\VerbatimInput{../jk_latch.v}

Временная диаграмма для JK-триггера изображена на рис.\ref{fig:jk_wave}.

\begin{figure}[H]
	\centering
	\includegraphics[width=0.95\linewidth]{imgs/jk_wave}
	\caption{Временная диаграмма для JK-триггера}
	\label{fig:jk_wave}
\end{figure}


\subsection{T-триггер}

%\VerbatimInput{../t_latch.v}

\begin{table}[]
	\begin{center}
		\begin{flushleft}
			\tablecaption{Таблица переходов для T-триггера}
		\end{flushleft}
		\label{tab:t_flip_flop}
		\begin{tabular}{|c|c|}
			\hline
			$T$ & $Q$                   \\ \hline
			0   & $Q_{prev}$            \\ \hline
			1   & $\overline{Q_{prev}}$ \\ \hline
		\end{tabular}
	\end{center}
\end{table}

\subsection{Измененный D-триггер}

%\VerbatimInput{../d_latch.v}

% Сравнение RTL

\subsection{Задний D-триггер}

%\VerbatimInput{../d_latch.v}

% Сравнение RTL

\subsection{Тестирование D-триггер}

\subsection{Тестирование JK-триггер}

\subsection{D-триггер с очисткой}

% Сравнение RTL

\subsection{D-триггер со сбросом}

% Сравнение RTL

\subsection{Какая то ересь в №10}

% Сравнение RTL

\subsection{Какая то ересь в №11}

% Сравнение RTL

\section{Контрольные вопросы}

\begin{enumerate}
	\item b) D-тpиrrep;
	\item AZ, BY, CX;
	\item Задержка распространения: $t_{pd}$ = максимальная задержка от входа к выходу;
	\item Синхронные триггеры позволяют изменять значение своего состояния только при подаче специального сигнала.
	Асинхронные могут изменять свое состояние в любой момент времени;
	\item Запрещенная комбинация -- комбинация входных сигналов, подача которых на вход триггера приведет его в неопределенное состояние;
	При подаче запрещенной комбинации на обоих выходах триггера будет низкий логический сигнал, при чем триггер установится в определенное состояние при подаче другого сигнала;
	\item D-триггер синхронный, в отличии от D-защелки;
	\item В чем отличие схемы мастер-помощник от схемы с использованием бистабильных ячеек при построении О-триггера?
	\item Что означает параметр время установки для последовательностных устройств?
	\item Что означает параметр время удержания для последовательностных устройств?
	\item Почему JК-триrгер называют универсальным триггером?
	\item Разработайте схему Т-триггера на основе О-триггера .
	\item Опишите особенности применения оператора непрерывного присваивания в языке Verilog.
	\item В чем отличие блокирующего присваивания от неблокирующего в языке Verilog?
	\item В чем отличие синхронного сброса от асинхронного? Приведите пример реализации	синхронного сброса на языке Verilog.
	\item Опишите назначение и примеры применения ключевого слова parameter в языке	Verilog.
	\item От чего зависит величина мощности, потребляемой логическим элементом в динамическом режиме?
	\item Что такое ячейка блокировки? Для чего она применяется?
	\item В чем состоит проблема использования блокировки тактового сигнала (gated clock)?
\end{enumerate}

\section{Выводы по работе}

В ходе работы получен опыт проектирования схем в программе Quartus с помощью языка Verilog.
Полученное устройство было протестировано с помощью бенчтестов в программе Quartus Simulation Waveform editor.
В процессе работы были смоделированы различные триггеры с синхронной и асинхронным управлением, сигналами сброса.
В процессе был получен опыт работы с платой DE10-Lite, на которой проверялась работоспособность полученного устройства.

\newpage 
\renewcommand{\refname}{{\normalsize Список использованных источников}} 
\centering 
\begin{thebibliography}{9} 
	\addcontentsline{toc}{section}{\refname} 
	\bibitem{Verilog} Thomas D., Moorby P. The Verilog Hardware Description Language. – Springer Science \& Business Media, 2008.
	\bibitem{citekey} Khor W. Y. et al. Evaluation of FPGA Based QSPI Flash Access Using Partial Reconfiguration //2019 7th International Conference on Smart Computing \& Communications (ICSCC). – IEEE, 2019. – С. 1-5
\end{thebibliography}

\end{document} % конец документа






