%!TEX TS-program = xelatex

% Шаблон документа LaTeX создан в 2018 году
% Алексеем Подчезерцевым
% В качестве исходных использованы шаблоны
% 	Данилом Фёдоровых (danil@fedorovykh.ru) 
%		https://www.writelatex.com/coursera/latex/5.2.2
%	LaTeX-шаблон для русской кандидатской диссертации и её автореферата.
%		https://github.com/AndreyAkinshin/Russian-Phd-LaTeX-Dissertation-Template

\documentclass[a4paper,14pt]{article}

\input{data/preambular.tex}
\begin{document} % конец преамбулы, начало документа
	\begin{titlepage}
	\begin{center}
		ФЕДЕРАЛЬНОЕ  ГОСУДАРСТВЕННОЕ АВТОНОМНОЕ \\
		ОБРАЗОВАТЕЛЬНОЕ УЧРЕЖДЕНИЕ ВЫСШЕГО ОБРАЗОВАНИЯ\\
		«НАЦИОНАЛЬНЫЙ ИССЛЕДОВАТЕЛЬСКИЙ УНИВЕРСИТЕТ\\
		«ВЫСШАЯ ШКОЛА ЭКОНОМИКИ»
	\end{center}
	
	\begin{center}
		\textbf{Московский институт электроники и математики}
		
		\textbf{Им. А.Н.Тихонова НИУ ВШЭ}
		
		\textbf{Департамент компьютерной инженерии}
	\end{center}
	\vspace{1ex}	
	\begin{center}
		Подчезерцев Алексей Евгеньевич, группа БИВ174
		
		Солодянкин Андрей Александрович, группа БИВ174
	\end{center}	
	\vspace{1ex}
	\begin{center}
		\textbf{ОТЧЕТ\\
		ПО ЛАБОРАТОРНОЙ РАБОТЕ №1
	}
	\end{center}	
	\vspace{2ex}
	\begin{center}
		по дисциплине «Проектирование систем на кристалле»
	\end{center}

	\vfill
	\begin{center}
		Москва \the\year \, г.
	\end{center}
\end{titlepage}
	\tableofcontents
	\pagebreak
	\section{Задание}
	
	\begin{enumerate}
		\item Разработайте RТL-модель и тест для рассмотренного счетчика . При возникновении
		сложностей обратитесь к дополнительным материалам к данной главе. Сравните полученные
		результаты.
		
		\item Проведите симуляцию модуля clk\_div, файлы с описанием которого находятся в
		дополнительных материалах к данной практической работе. Определите частоту выходного
		сигнала clk\_out. Измените код так, чтобы скважность сигнала составляла 20 %, 50 %. Опишите
		полученные результаты.
		
		\item Проанализируйте работу модуля pwm из дополнительных материалов к данной практической
		работе . Определите, в каких случаях светодиод будет светить ярче. Определите скважность
		выходного сигнала pwm\_out для каждого случая. Проанализируйте временные диаграммы на
		Рисунке 6.12. Разработайте собственный модуль ШИМ. Сравните полученные результаты.
		
		\item Разработайте RТL-модель и тест для счетчика Грея. При возникновении сложностей
		обратитесь к дополнительным материалам к данной главе. Проанализируйте полученные
		результаты.
		
		\item Разработайте RТL-модель и тест сдвигового регистра, который бы соответствовал
		приведенному примеру. При возникновении сложностей обратитесь к дополнительным
		материалам к данной главе. Проанализируйте полученные результаты .
		
	\end{enumerate}
	
	\section{Дополнительные задания}
	
	\subsection{Задание 1}
	
	Был разработан счетчик размерностью 8 бит (коэффициент пересчета 256).
	Результат моделирования на рис. \ref{fig:z1_rtl}. Результат тестирования на рис. \ref{fig:z1_test}.
	 
	\begin{figure}[H]
		\centering
		\includegraphics[width=\linewidth]{images/z1_rtl}
		\caption{RTL-схема для обычного счетчика}
		\label{fig:z1_rtl}
	\end{figure}

	\begin{figure}[H]
		\centering
		\includegraphics[width=\linewidth]{images/z1_test}
		\caption{Результат тестирования обычного счетчика}
		\label{fig:z1_test}
	\end{figure}

	Был разработан счетчик такой же размерности с возможностью подсчета в положительную и отрицательную сторону, в зависимости от сигнала.
	Результат моделирования на рис. \ref{fig:z2_rtl}. Результат тестирования на рис. \ref{fig:z2_test}.
	
	\begin{figure}[H]
		\centering
		\includegraphics[width=\linewidth]{images/z2_rtl}
		\caption{RTL-схема для счетчика с направлением подсчета}
		\label{fig:z2_rtl}
	\end{figure}
	
	\begin{figure}[H]
		\centering
		\includegraphics[width=\linewidth]{images/z2_test}
		\caption{Результат тестирования для счетчика с направлением подсчета}
		\label{fig:z2_test}
	\end{figure}

	\subsection{Задание 2}
	
	Исходный код модуля понижал частоту в $2^{24} = 16777216$ раз.
	Такое изменение трудно заметить на симуляции, поэтому код модуля был изменен для понижения частоты в $2^8 = 256$ раз.
	
	Исходный код модуля.
	
	\VerbatimInput{../z1/clk_divider.v}
	
	В оригинальном модуле коэффициент заполнения был равен 50\%.
	Действительно, выходной сигнал был активен в половине случаев.
	Для понижения коэффициента заполнения до 20\% было вычислено значение, начиная с которого необходимо выводить высокий уровень сигнала.
	Результат симуляции на рис. \ref{fig:z3}.
	
	\begin{figure}[H]
		\centering
		\includegraphics[width=\linewidth]{images/z3}
		\caption{Результат тестирования понижателя частоты}
		\label{fig:z3}
	\end{figure}

	\subsection{Задание 3}

	Чем больше коэффициент заполнения на сигнале ШИМ, тем ярче горит светодиод.
	Пример тестирования представлен на рис. \ref{fig:z4}.
	В результате тестирования яркость светодиода будет постепенно возрастать.
	
	\begin{figure}[H]
		\centering
		\includegraphics[width=\linewidth]{images/z4}
		\caption{Результат тестирования ШИМ}
		\label{fig:z4}
	\end{figure}

	\subsection{Задание 4}
	
	Был разработан счетчик Грея.
	При положительном фронте $clk$ происходит увеличение счетчика на 1 в коде Грея При этом сигнал $enable$ должен быть 1, в противном случае состояние счетчика не изменится. При подаче сигнала $rst\_n$ равного 0 происходит обнуление счетчика.
	Результат моделирования на рис. \ref{fig:z5_rtl}. Результат тестирования на рис. \ref{fig:z5_test}.
	
	\begin{figure}[H]
		\centering
		\includegraphics[width=\linewidth]{images/z5_rtl}
		\caption{RTL-схема для счетчика Грея}
		\label{fig:z5_rtl}
	\end{figure}
	
	\begin{figure}[H]
		\centering
		\includegraphics[width=\linewidth]{images/z5_test}
		\caption{Результат тестирования для счетчика Грея}
		\label{fig:z5_test}
	\end{figure}

	\subsection{Задание 5}
	
	\section{Задания для самостоятельной работы}
	
	В качестве основы для выполнения задач необходимо использовать проекты из
	Разделов 6.5.1 и 6.6.1. Для демонстрации значений счетчика и сдвигового регистра
	используйте семисегментный дисплей и светодиоды, которые расположены на отладочной
	плате с ПЛИС.
	
	Измените проект счетчика так, чтобы цифры двух произвольных разрядов
	инкрементировались, а цифры других разрядов декрементировались .
			
	\section{Контрольные вопросы}
	
	\begin{enumerate}
		\item В чем основные отличия последовательностных устройств от комбинационных?
		
		\item Почему сложно разрабатывать асинхронные схемы, а в современной электронике
		большие асинхронные цифровые схемы практически не используются?
		
		\item Опишите модель Хаффмана для последовательностных устройств .
		
		\item Каково различие между блокирующим и неблокирующим присвоением?
		
		\item К чему приведет изменение порядка операций в блоке always?
		
		\item В каких случаях следует использовать блокирующее присвоение, а в каких - неблокирующее?
		
		\item Опишите порядок выполнения циклов симуляции, определенный в IEEE Verilog	Standart?
		
		\item В результате чего при синтезе возникают защелки (latch), и чем они опасны?
		
		\item Для чего нужны счетчики? Какими они бывают? Опишите на Verilog пример простого
		счетчика.
		
		\item Для каких задач используются делители частоты?
			
		\item Опишите, что такое широтно-импульсная модуляция (ШИМ), и для чего она
		применяется.
		
		\item Как строится счетчик Грея? В чем его особенность? Опишите пример счетчика Грея на
		Verilog.
		
		\item Для чего нужны сдвиговые регистры . Какими бывают виды сдвиговых регистров?
		Приведите пример сдвигового регистра на Verilog.
		
		\item Что такое циклический избыточный код (CRC)? Приведите области применения .
		
		\item Приведите примеры периферийных устройств, используемых с отладочными платами
		ПЛИС; опишите процесс их подключения.
	\end{enumerate}
	
	\section{Выводы по работе}
	
	В ходе работы получен опыт проектирования схем в программе Quartus с помощью языка Verilog.
	Полученное устройство было протестировано с помощью бенчтестов в программе Quartus Simulation Waveform editor и ModelSim.
	В процессе работы были смоделированы счетчики, сдвиговые регистры, ШИМ и счетчик Грея.
	%В процессе работы были смоделированы различные шифраторы и дешифраторы, протестированы способы оптимизации схемы, а так же рассчитаны временные параметры схемы с различными способами оптимизации.
	В процессе был получен опыт работы с платой DE10-Lite, на которой проверялась работоспособность полученного устройства.
	
	\newpage 
	\renewcommand{\refname}{{\normalsize Список использованных источников}} 
	\centering 
	\begin{thebibliography}{9} 
		\addcontentsline{toc}{section}{\refname} 
		\bibitem{Verilog} Thomas D., Moorby P. The Verilog Hardware Description Language. – Springer Science \& Business Media, 2008.
		\bibitem{citekey} Khor W. Y. et al. Evaluation of FPGA Based QSPI Flash Access Using Partial Reconfiguration //2019 7th International Conference on Smart Computing \& Communications (ICSCC). – IEEE, 2019. – С. 1-5
	\end{thebibliography}
	
\end{document} % конец документа
