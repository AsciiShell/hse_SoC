%!TEX TS-program = xelatex

% Шаблон документа LaTeX создан в 2018 году
% Алексеем Подчезерцевым
% В качестве исходных использованы шаблоны
% 	Данилом Фёдоровых (danil@fedorovykh.ru) 
%		https://www.writelatex.com/coursera/latex/5.2.2
%	LaTeX-шаблон для русской кандидатской диссертации и её автореферата.
%		https://github.com/AndreyAkinshin/Russian-Phd-LaTeX-Dissertation-Template

\documentclass[a4paper,14pt]{article}

\input{data/preambular.tex}
\begin{document} % конец преамбулы, начало документа
\begin{titlepage}
	\begin{center}
		ФЕДЕРАЛЬНОЕ  ГОСУДАРСТВЕННОЕ АВТОНОМНОЕ \\
		ОБРАЗОВАТЕЛЬНОЕ УЧРЕЖДЕНИЕ ВЫСШЕГО ОБРАЗОВАНИЯ\\
		«НАЦИОНАЛЬНЫЙ ИССЛЕДОВАТЕЛЬСКИЙ УНИВЕРСИТЕТ\\
		«ВЫСШАЯ ШКОЛА ЭКОНОМИКИ»
	\end{center}
	
	\begin{center}
		\textbf{Московский институт электроники и математики}
		
		\textbf{Им. А.Н.Тихонова НИУ ВШЭ}
		
		\textbf{Департамент компьютерной инженерии}
	\end{center}
	\vspace{1ex}	
	\begin{center}
		Подчезерцев Алексей Евгеньевич, группа БИВ174
		
		Солодянкин Андрей Александрович, группа БИВ174
	\end{center}	
	\vspace{1ex}
	\begin{center}
		\textbf{ОТЧЕТ\\
		ПО ЛАБОРАТОРНОЙ РАБОТЕ №1
	}
	\end{center}	
	\vspace{2ex}
	\begin{center}
		по дисциплине «Проектирование систем на кристалле»
	\end{center}

	\vfill
	\begin{center}
		Москва \the\year \, г.
	\end{center}
\end{titlepage}
\tableofcontents
\pagebreak

\section{Задание}

Создать одноразрядный и двуразрядный сумматор, создать вычитатель на базе сумматора.

\section{Выполнение работы}

Составим схему одноразрядного двоичного сумматора  (рис.~\ref{fig:schema_1}).
Данная схема принимает по биту для каждого числа и знак переноса, на выходе s получаем результат суммирования, на выходе $p_{out}$ -- перенос.
Результат тестирования и временные задержки представлены на рис.~\ref{fig:wave_1} и~\ref{fig:time_1} соответственно.

\begin{figure}[H]
	\centering
	\includegraphics[width=\linewidth]{image/schema_1}
	\caption{Схема одноразрядного сумматора}
	\label{fig:schema_1}
\end{figure}

\begin{figure}[H]
	\centering
	\includegraphics[width=0.5\linewidth]{image/wave_1}
	\caption{Временные диаграмма одноразрядного сумматора}
	\label{fig:wave_1}
\end{figure}

\begin{figure}[H]
	\centering
	\includegraphics[width=0.7\linewidth]{image/time_1}
	\caption{Временные задержки одноразрядного сумматора}
	\label{fig:time_1}
\end{figure}

Далее данную схему можно обернуть в один блок для упрощения разработки сложных схем.
На рис.~\ref{fig:schema_2} изображена схема сумматора с применением данного блока. 
Сигнал переноса передается последовательно от одного сумматора к следующему.
Результат тестирования и временные задержки представлены на рис.~\ref{fig:wave_2} и~\ref{fig:time_2} соответственно.

\begin{figure}[H]
	\centering
	\includegraphics[width=\linewidth]{image/schema_2}
	\caption{Схема двуразрядного сумматора}
	\label{fig:schema_2}
\end{figure}

\begin{figure}[H]
	\centering
	\includegraphics[width=0.5\linewidth]{image/wave_2}
	\caption{Тестирование двуразрядного сумматора}
	\label{fig:wave_2}
\end{figure}

\begin{figure}[H]
	\centering
	\includegraphics[width=0.7\linewidth]{image/time_2}
	\caption{Временные задержки двуразрядного сумматора}
	\label{fig:time_2}
\end{figure}

Аналогично был разработан четырехразрядный сумматор (рис. \ref{fig:schema_4}).
Результат тестирования и временные задержки представлены на рис.~\ref{fig:wave_4} и~\ref{fig:time_4} соответственно.

\begin{figure}[H]
	\centering
	\includegraphics[width=\linewidth]{image/schema_4}
	\caption{Схема четырехразрядного сумматора}
	\label{fig:schema_4}
\end{figure}

\begin{figure}[H]
	\centering
	\includegraphics[width=\linewidth]{image/wave_4}
	\caption{Тестирование четырехразрядного сумматора}
	\label{fig:wave_4}
\end{figure}

\begin{figure}[H]
	\centering
	\includegraphics[width=0.6\linewidth]{image/time_4}
	\caption{Временные задержки четырехразрядного сумматора}
	\label{fig:time_4}
\end{figure}

Для работы вычитателя можно взять сумматор и инвертировать второе число.
Полученная схема изображена на рис.~\ref{fig:schema_-}.
Результат тестирования и временные задержки представлены на рис.~\ref{fig:wave_-} и~\ref{fig:time_-} соответственно.

\begin{figure}[H]
	\centering
	\includegraphics[width=\linewidth]{image/schema_-}
	\caption{Схема двуразрядного вычитателя}
	\label{fig:schema_-}
\end{figure}

\begin{figure}[H]
	\centering
	\includegraphics[width=0.5\linewidth]{image/wave_-}
	\caption{Тестирование двуразрядного вычитателя}
	\label{fig:wave_-}
\end{figure}

\begin{figure}[H]
	\centering
	\includegraphics[width=0.7\linewidth]{image/time_-}
	\caption{Временные задержки двуразрядного вычитателя}
	\label{fig:time_-}
\end{figure}

Было произведено тестирование работы полученного устройства на плате.
В данном примере (рис.~\ref{fig:demo}) из числа 1 (LEDR1, LEDR0) вычитается число 3 (LEDR3, LEDR2), индикатор SW0 говорит, что число отрицательное, а SW2 и SW1 показывает число -2 в дополнительном коде.

\begin{figure}[H]
	\centering
	\includegraphics[width=0.8\linewidth]{image/demo}
	\caption{Результат работы двуразрядного вычитателя}
	\label{fig:demo}
\end{figure}

\section{Выводы по работе}

В ходе работы был получен блок одинарного сумматора, который позволяет создавать сумматоры любой размерности. 
Так же данное устройство способно выполнять функции вычитателя путем сложения заданного числа с инверсией второго слагаемого. 
Разработанная схема была протестирована с помощью Waveform на соответствие полученного и ожидаемого результата.
Итоговый вариант был проверен на плате DE10-Lite, наблюдения подтвердили работоспособность устройства и правильность работы.

\newpage 
\renewcommand{\refname}{{\normalsize Список использованных источников}} 
\centering 
\begin{thebibliography}{9} 
	\addcontentsline{toc}{section}{\refname} 
	\bibitem{Verilog} Thomas D., Moorby P. The Verilog Hardware Description Language. – Springer Science \& Business Media, 2008.
	\bibitem{Quartus} Антонов А., Филиппов А., Золотухо Р. Средства системной отладки САПР Quartus II //Компоненты и технологии. – 2008. – №. 89.
\end{thebibliography}

\end{document} % конец документа