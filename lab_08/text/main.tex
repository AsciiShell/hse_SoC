%!TEX TS-program = xelatex

% Шаблон документа LaTeX создан в 2018 году
% Алексеем Подчезерцевым
% В качестве исходных использованы шаблоны
% 	Данилом Фёдоровых (danil@fedorovykh.ru) 
%		https://www.writelatex.com/coursera/latex/5.2.2
%	LaTeX-шаблон для русской кандидатской диссертации и её автореферата.
%		https://github.com/AndreyAkinshin/Russian-Phd-LaTeX-Dissertation-Template

\documentclass[a4paper,14pt]{article}

\input{data/preambular.tex}
\begin{document} % конец преамбулы, начало документа
	\begin{titlepage}
	\begin{center}
		ФЕДЕРАЛЬНОЕ  ГОСУДАРСТВЕННОЕ АВТОНОМНОЕ \\
		ОБРАЗОВАТЕЛЬНОЕ УЧРЕЖДЕНИЕ ВЫСШЕГО ОБРАЗОВАНИЯ\\
		«НАЦИОНАЛЬНЫЙ ИССЛЕДОВАТЕЛЬСКИЙ УНИВЕРСИТЕТ\\
		«ВЫСШАЯ ШКОЛА ЭКОНОМИКИ»
	\end{center}
	
	\begin{center}
		\textbf{Московский институт электроники и математики}
		
		\textbf{Им. А.Н.Тихонова НИУ ВШЭ}
		
		\textbf{Департамент компьютерной инженерии}
	\end{center}
	\vspace{1ex}	
	\begin{center}
		Подчезерцев Алексей Евгеньевич, группа БИВ174
		
		Солодянкин Андрей Александрович, группа БИВ174
	\end{center}	
	\vspace{1ex}
	\begin{center}
		\textbf{ОТЧЕТ\\
		ПО ЛАБОРАТОРНОЙ РАБОТЕ №1
	}
	\end{center}	
	\vspace{2ex}
	\begin{center}
		по дисциплине «Проектирование систем на кристалле»
	\end{center}

	\vfill
	\begin{center}
		Москва \the\year \, г.
	\end{center}
\end{titlepage}
	\tableofcontents
	\pagebreak
	\section{Задание}
	
	\begin{enumerate}
		\item В приведенной выше реализации автомата Мура задайте различные способы кодирования
		состояний и сравните результаты компиляции.
		
		\item В приведенной выше реализации автомата Мили задайте различные способы кодирования
		состояний и сравните результаты компиляции.
	\end{enumerate}
	
	\section{Дополнительные задания}
	
	\subsection{Задание 1}
	
	Автомат Мура был собран со следующими способами кодирования: последовательный, one hot encoding, код Грея.
	Код Джонсона для 3 состояний совпадает с кодом Грея.
	Результат компиляции, а так же RTL диаграмма совпали для каждого варианта.
	Результаты для последовательного кодирования представлены на рис. \ref{fig:z1_bin_report} и \ref{fig:z1_bin_rtl}
	
	\begin{figure}[H]
		\centering
		\includegraphics[width=0.8\linewidth]{images/z1_bin_report}
		\caption{Результат компиляции автомата Мура с последовательным кодированием состояния}
		\label{fig:z1_bin_report}
	\end{figure}	
	
	\begin{figure}[H]
		\centering
		\includegraphics[width=\linewidth]{images/z1_bin_rtl}
		\caption{RTL-схема автомата Мура с последовательным кодированием состояния}
		\label{fig:z1_bin_rtl}
	\end{figure}

	\subsection{Задание 2}
	
	Автомат Мили был собран со следующими способами кодирования: one hot encoding, код Грея.
	Как и в задании 1, результат компиляции, а так же RTL диаграмма совпали для каждого варианта.
	Это можно объяснить тем, что представление кода хранится и обрабатывается в блоке $state$, его состояние не доступно снаружи схемы.
	На небольших примерах способы кодирования не сильно отличаются размерностью, поэтому нет отличий в результатах компиляции по используемым ресурсам.
	Результаты для последовательного кодирования представлены на рис. \ref{fig:z2_gray_report} и \ref{fig:z2_gray_rtl}
	
	\begin{figure}[H]
		\centering
		\includegraphics[width=0.8\linewidth]{images/z2_gray_report}
		\caption{Результат компиляции автомата Мили с кодированием состояния кодом Грея}
		\label{fig:z2_gray_report}
	\end{figure}	
	
	\begin{figure}[H]
		\centering
		\includegraphics[width=\linewidth]{images/z2_gray_rtl}
		\caption{RTL-схема автомата Мили с кодированием состояния кодом Грея}
		\label{fig:z2_gray_rtl}
	\end{figure}

	\section{Задания для самостоятельной работы}

			
	\section{Контрольные вопросы}
	
	\begin{enumerate}
		\item 	
		
	\end{enumerate}
	
	\section{Выводы по работе}
	
	В ходе работы получен опыт проектирования схем в программе Quartus с помощью языка Verilog.
	Полученное устройство было протестировано с помощью бенчтестов в программе Quartus Simulation Waveform editor и ModelSim.
	В процессе работы были смоделированы запоминающие устройства, очередь, стек.
	%В процессе работы были смоделированы различные шифраторы и дешифраторы, протестированы способы оптимизации схемы, а так же рассчитаны временные параметры схемы с различными способами оптимизации.
	В процессе был получен опыт работы с платой DE10-Lite, на которой проверялась работоспособность полученного устройства.
	
	\newpage 
	\renewcommand{\refname}{{\normalsize Список использованных источников}} 
	\centering 
	\begin{thebibliography}{9} 
		\addcontentsline{toc}{section}{\refname} 
		\bibitem{Verilog} Thomas D., Moorby P. The Verilog Hardware Description Language. – Springer Science \& Business Media, 2008.
		\bibitem{citekey} Khor W. Y. et al. Evaluation of FPGA Based QSPI Flash Access Using Partial Reconfiguration //2019 7th International Conference on Smart Computing \& Communications (ICSCC). – IEEE, 2019. – С. 1-5
	\end{thebibliography}
	
\end{document} % конец документа
