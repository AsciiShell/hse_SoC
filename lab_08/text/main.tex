%!TEX TS-program = xelatex

% Шаблон документа LaTeX создан в 2018 году
% Алексеем Подчезерцевым
% В качестве исходных использованы шаблоны
% 	Данилом Фёдоровых (danil@fedorovykh.ru) 
%		https://www.writelatex.com/coursera/latex/5.2.2
%	LaTeX-шаблон для русской кандидатской диссертации и её автореферата.
%		https://github.com/AndreyAkinshin/Russian-Phd-LaTeX-Dissertation-Template

\documentclass[a4paper,14pt]{article}

\input{data/preambular.tex}
\begin{document} % конец преамбулы, начало документа
	\begin{titlepage}
	\begin{center}
		ФЕДЕРАЛЬНОЕ  ГОСУДАРСТВЕННОЕ АВТОНОМНОЕ \\
		ОБРАЗОВАТЕЛЬНОЕ УЧРЕЖДЕНИЕ ВЫСШЕГО ОБРАЗОВАНИЯ\\
		«НАЦИОНАЛЬНЫЙ ИССЛЕДОВАТЕЛЬСКИЙ УНИВЕРСИТЕТ\\
		«ВЫСШАЯ ШКОЛА ЭКОНОМИКИ»
	\end{center}
	
	\begin{center}
		\textbf{Московский институт электроники и математики}
		
		\textbf{Им. А.Н.Тихонова НИУ ВШЭ}
		
		\textbf{Департамент компьютерной инженерии}
	\end{center}
	\vspace{1ex}	
	\begin{center}
		Подчезерцев Алексей Евгеньевич, группа БИВ174
		
		Солодянкин Андрей Александрович, группа БИВ174
	\end{center}	
	\vspace{1ex}
	\begin{center}
		\textbf{ОТЧЕТ\\
		ПО ЛАБОРАТОРНОЙ РАБОТЕ №1
	}
	\end{center}	
	\vspace{2ex}
	\begin{center}
		по дисциплине «Проектирование систем на кристалле»
	\end{center}

	\vfill
	\begin{center}
		Москва \the\year \, г.
	\end{center}
\end{titlepage}
	\tableofcontents
	\pagebreak
	\section{Задание}
	
	\begin{enumerate}
		\item В приведенной выше реализации автомата Мура задайте различные способы кодирования
		состояний и сравните результаты компиляции.
		
		\item В приведенной выше реализации автомата Мили задайте различные способы кодирования
		состояний и сравните результаты компиляции.
	\end{enumerate}
	
	\section{Дополнительные задания}
	
	\subsection{Задание 1}
	
	Автомат Мура был собран со следующими способами кодирования: последовательный, one hot encoding, код Грея.
	Код Джонсона для 3 состояний совпадает с кодом Грея.
	Результат компиляции, а так же RTL диаграмма совпали для каждого варианта.
	Результаты для последовательного кодирования представлены на рис. \ref{fig:z1_bin_report} и \ref{fig:z1_bin_rtl}
	
	\begin{figure}[H]
		\centering
		\includegraphics[width=0.8\linewidth]{images/z1_bin_report}
		\caption{Результат компиляции автомата Мура с последовательным кодированием состояния}
		\label{fig:z1_bin_report}
	\end{figure}	
	
	\begin{figure}[H]
		\centering
		\includegraphics[width=\linewidth]{images/z1_bin_rtl}
		\caption{RTL-схема автомата Мура с последовательным кодированием состояния}
		\label{fig:z1_bin_rtl}
	\end{figure}

	\subsection{Задание 2}
	
	Автомат Мили был собран со следующими способами кодирования: one hot encoding, код Грея.
	Как и в задании 1, результат компиляции, а так же RTL диаграмма совпали для каждого варианта.
	Это можно объяснить тем, что представление кода хранится и обрабатывается в блоке $state$, его состояние не доступно снаружи схемы.
	На небольших примерах способы кодирования не сильно отличаются размерностью, поэтому нет отличий в результатах компиляции по используемым ресурсам.
	Результаты для последовательного кодирования представлены на рис. \ref{fig:z2_gray_report} и \ref{fig:z2_gray_rtl}
	
	\begin{figure}[H]
		\centering
		\includegraphics[width=0.8\linewidth]{images/z2_gray_report}
		\caption{Результат компиляции автомата Мили с кодированием состояния кодом Грея}
		\label{fig:z2_gray_report}
	\end{figure}	
	
	\begin{figure}[H]
		\centering
		\includegraphics[width=\linewidth]{images/z2_gray_rtl}
		\caption{RTL-схема автомата Мили с кодированием состояния кодом Грея}
		\label{fig:z2_gray_rtl}
	\end{figure}

	\section{Задания для самостоятельной работы}

	\subsection{Задание 1}
	
	В соответствии с таблицей своего варианта (рис. \ref{fig:extra_table}) постройте граф конечного автомата.
	Разработайте код конечного автомата с комбинационными выходами и синхронными
	выходами. Выполните компиляцию проекта и синтезируйте RТL-представление проекта.
	Проведите моделирование конечного автомата
	
	\begin{figure}[H]
		\centering
		\includegraphics[width=\linewidth]{images/extra_table}
		\caption{Дополнительное задание}
		\label{fig:extra_table}
	\end{figure}

	Исходный код конечного автомата:
	\VerbatimInput{../extra/extra.v}
	
	RTL-диаграмма и State Machine View доступны на рис. \ref{fig:extra_rtl} и \ref{fig:extra_machine} соответственно.
	По результатам наблюдений, граф конечного автомата совпал с заранее созданным графом.
	Результат симуляции изображен на рис. \ref{fig:extra_sim}
	
	\begin{figure}[H]
		\centering
		\includegraphics[width=\linewidth]{images/extra_rtl}
		\caption{RTL-диаграмма для конечного автомата}
		\label{fig:extra_rtl}
	\end{figure}

	\begin{figure}[H]
		\centering
		\includegraphics[width=\linewidth]{images/extra_machine}
		\caption{State Machine Viewа для конечного автомата}
		\label{fig:extra_machine}
	\end{figure}

	\begin{figure}[H]
		\centering
		\includegraphics[width=\linewidth]{images/extra_sim}
		\caption{Результат симуляции для конечного автомата}
		\label{fig:extra_sim}
	\end{figure}

	\subsection{Задание 2}
	
	Напишите возле каждой диаграммы состояния (Рисунок 8.22) имя соответствующего модуля из Листинга 8.9.
	
	Решение на рис. \ref{fig:extra_2}
	
	\begin{figure}[H]
		\centering
		\includegraphics[width=\linewidth]{images/extra_2}
		\caption{Решение задания 2}
		\label{fig:extra_2}
	\end{figure}
	
	
	\subsection{Задание 3}
	
	Напишите возле каждого рисунка (Рисунок 8.23) имя соответствующего модуля из Листинга 8.10.
	
	Решение на рис. \ref{fig:extra_3}
	
	\begin{figure}[H]
		\centering
		\includegraphics[width=\linewidth]{images/extra_3}
		\caption{Решение задания 3}
		\label{fig:extra_3}
	\end{figure}

	\section{Контрольные вопросы}
	
	\begin{enumerate}
		\item Приведите схему конечного автомата Мура.
		
		В автомате Мура выходной сигнал зависит только от текущего состояния системы (рис. \ref{fig:q1}).
		
		\begin{figure}[H]
			\centering
			\includegraphics[width=\linewidth]{images/q1}
			\caption{Схема автомата Мура}
			\label{fig:q1}
		\end{figure}
	
		\item Приведите пример описания конечного автомата Мура на Verilog.
		
		Исходный код конечного автомата Мура с кодированием состояния методом one hot encoding:
		\VerbatimInput{../z1/lab8_ohe.v}
		
		\item Приведите пример описания конечного автомата Мили на Verilog.
		
		Исходный код конечного автомата Мили с кодированием состояния методом one hot encoding:
		\VerbatimInput{../z2/lab8_ohe.v}
		
		\item Как происходит кодирование конечных автоматов в State Machine Viewer Quartus
		Prime?
		
		Каждое состояние описывается кругом, направления возможных переходов показаны стрелками.
		Вход в граф показан отдельной стрелкой, что означает положение по-умолчанию.
		
		\item Приведите схему конечного автомата Мили.
		
		В автомате Мили выходной сигнал зависит не только от текущего состояния системы, но и от входных сигналов (рис. \ref{fig:q5}).
		
		\begin{figure}[H]
			\centering
			\includegraphics[width=\linewidth]{images/q5}
			\caption{Схема автомата Мили}
			\label{fig:q5}
		\end{figure}
		
		\item Приведите примеры методов кодирования конечных автоматов. Охарактеризуйте их.
		
		Существует несколько способов кодирования состояния конечных автоматов:
		
		\begin{enumerate}
			\item one hot -- все биты, кроме одного, равны 0; единственный ненулевой бит обозначает порядковый номер состояния;
			
			\item Sequential -- каждое состояние кодируется порядковым номером в битовом представлении;
			
			\item Gray -- каждое состояние кодируется кодом Грея;
			
			\item Johnson -- каждое состояние кодируется кодом Джонсона, аналогичному коду Грея, но требующему меньше логики.
		\end{enumerate}
	
		Кроме того, пользователь может сам закодировать состояния конечного автомата любой желаемой константой.
		
		Достоинство первого способа в том, что сигнал уже декодирован и не требует сложной логики для работы. С другой стороны, метод one-hot требует N бит памяти, что для автоматов с большим количеством состояний будет не самым эффективным решением по памяти.
		
	\end{enumerate}
	
	\section{Выводы по работе}
	
	В ходе работы получен опыт проектирования схем в программе Quartus с помощью языка Verilog.
	Полученное устройство было протестировано с помощью бенчтестов в программе Quartus Simulation Waveform editor.
	В процессе работы были смоделированы конечные автоматы Мура и Мили.
	%В процессе работы были смоделированы различные шифраторы и дешифраторы, протестированы способы оптимизации схемы, а так же рассчитаны временные параметры схемы с различными способами оптимизации.
	В процессе был получен опыт работы с платой DE10-Lite, на которой проверялась работоспособность полученного устройства.
	
	\newpage 
	\renewcommand{\refname}{{\normalsize Список использованных источников}} 
	\centering 
	\begin{thebibliography}{9} 
		\addcontentsline{toc}{section}{\refname} 
		\bibitem{Verilog} Thomas D., Moorby P. The Verilog Hardware Description Language. – Springer Science \& Business Media, 2008.
		\bibitem{citekey} Khor W. Y. et al. Evaluation of FPGA Based QSPI Flash Access Using Partial Reconfiguration //2019 7th International Conference on Smart Computing \& Communications (ICSCC). – IEEE, 2019. – С. 1-5
	\end{thebibliography}
	
\end{document} % конец документа
