%!TEX TS-program = xelatex

% Шаблон документа LaTeX создан в 2018 году
% Алексеем Подчезерцевым
% В качестве исходных использованы шаблоны
% 	Данилом Фёдоровых (danil@fedorovykh.ru) 
%		https://www.writelatex.com/coursera/latex/5.2.2
%	LaTeX-шаблон для русской кандидатской диссертации и её автореферата.
%		https://github.com/AndreyAkinshin/Russian-Phd-LaTeX-Dissertation-Template

\documentclass[a4paper,14pt]{article}

\input{data/preambular.tex}
\begin{document} % конец преамбулы, начало документа
\begin{titlepage}
	\begin{center}
		ФЕДЕРАЛЬНОЕ  ГОСУДАРСТВЕННОЕ АВТОНОМНОЕ \\
		ОБРАЗОВАТЕЛЬНОЕ УЧРЕЖДЕНИЕ ВЫСШЕГО ОБРАЗОВАНИЯ\\
		«НАЦИОНАЛЬНЫЙ ИССЛЕДОВАТЕЛЬСКИЙ УНИВЕРСИТЕТ\\
		«ВЫСШАЯ ШКОЛА ЭКОНОМИКИ»
	\end{center}
	
	\begin{center}
		\textbf{Московский институт электроники и математики}
		
		\textbf{Им. А.Н.Тихонова НИУ ВШЭ}
		
		\textbf{Департамент компьютерной инженерии}
	\end{center}
	\vspace{1ex}	
	\begin{center}
		Подчезерцев Алексей Евгеньевич, группа БИВ174
		
		Солодянкин Андрей Александрович, группа БИВ174
	\end{center}	
	\vspace{1ex}
	\begin{center}
		\textbf{ОТЧЕТ\\
		ПО ЛАБОРАТОРНОЙ РАБОТЕ №1
	}
	\end{center}	
	\vspace{2ex}
	\begin{center}
		по дисциплине «Проектирование систем на кристалле»
	\end{center}

	\vfill
	\begin{center}
		Москва \the\year \, г.
	\end{center}
\end{titlepage}
\tableofcontents
\pagebreak
\section{Задание}

\begin{enumerate}
	\item Разработайте принципиальную схему неприоритетного шифратора, таблица истинности
	которого приведена в Таблице 3.1. Используйте логические элементы И-НЕ, ИЛИ-НЕ;
	
	\item Исследуйте работу, синтезируйте и реализуйте на FPGA плате все приведенные в данном
	разделе примеры шифраторов;
	
	\item Исследуйте работу, синтезируйте и реализуйте на FPGA плате параметрический дешифратор
	с использованием оператора сдвига (Листинг 3.14);
	
	\item Синтезируйте описанные выше параметрические шифраторы и дешифраторы с различными
	настройками профилей оптимизации и сравните параметры полученных устройств;
	Опишите, как отличаются пути прохождения сигналов и чем это обусловлено;
	
	\item Разработайте конвертор унарного кода в код Грея.
\end{enumerate}

\section{Выполнение работы}

\subsection{Неприоритетный шифратор}
 
Была разработана схема неприоритетного шифратора в базисе И-НЕ, ИЛИ-НЕ (рис.\ref{fig:z1_schema}).
\begin{figure}[H]
	\centering
	\includegraphics[width=0.9\linewidth]{imgs/z1_schema}
	\caption{Схема неприоритетного шифратора}
	\label{fig:z1_schema}
\end{figure}

\subsection{Шифраторы}

Шифратор на основе оператора непрерывного присваивания.
\VerbatimInput{../z2/b1_enc_assign.v}

Шифратор на основе оператора if.
\VerbatimInput{../z2/b2_enc_if.v}

Шифратор на основе оператора case.
\VerbatimInput{../z2/b3_enc_case.v}

Приоритетный шифратор на основе оператора if.
\VerbatimInput{../z2/b4_pri_enc_if.v}

Приоритетный шифратор на основе оператора непрерывного присваивания.
\VerbatimInput{../z2/b5_pri_enc_assign.v}

Приоритетный шифратор на основе оператора сдвиговых регистров.
\VerbatimInput{../z2/b12_anybit_enc.v}

\subsection{Дешифраторы}

\VerbatimInput{../z3/7_lab3_hdl_4bit_dec_shift/lab3.v}

\subsection{Анализ параметров оптимизации}

\subsection{Конвертор унарного кода в код Грея}

\VerbatimInput{../gray_case.v}

Результаты симуляции изображен на рис.\ref{fig:gray_sim}.
\begin{figure}[H]
	\centering
	\includegraphics[width=0.9\linewidth]{imgs/gray_sim}
	\caption{Результат симуляции конвертера унарного кода в код Грея}
	\label{fig:gray_sim}
\end{figure}
 
\section{Контрольные вопросы}

\begin{enumerate}
	\item Опишите, что такое шифратор. 
	Каковы различия между приоритетным и неприоритетным шифраторами?
	
	Шифратор -- устройство, которое преобразует входной унарный код в двоичный меньшей разрядности.
	Приоритетный шифратор может обрабатывать не только унарный код, но и с несколькими логическими 1. 
	На выходе будет такой сигнал, который соответствует входному с наибольшим приоритетом.
	Неприоритетный такую работу не поддерживает.
	
	\item Опишите, как реализовать неприоритетный шифратор с помощью операторов assign,	if и case. 
	В чем их различия? 
	Какой способ лучше?
	
	Шифратор на операторе assign самый простой, однако требует аккуратности при проектировании.
	Шифратор на операторе if выглядит громоздко как и при проектировании, так и при генерации схемы.
	Шифратор на операторе case более простой с точки зрения написания кода и итоговой схемы.
	
	Однако по временным задержкам самый быстрый шифратор на операторе assign.
	
	\item Опишите, как реализовать приоритетный шифратор с помощью операторов assign и if. 
	В чем их различия? 
	Какой способ лучше?
	
	Приоритетный шифратор реализуется путем последовательной проверки каждого бита входного сигнала.
	Проверка реализуется через if или тернарный оператор и assign.
	Результат моделирования данных схем одинаковый.
	
	\item Что характеризуют такие параметры как задержка распространения ($t_{pd}$) и задержка реакции ($t_{cd}$)?
	
	Задержка распространения -- это максимальное время от начала изменения входного сигнала схемы до момента, когда	все ее выходы достигнут своих стационарных состояний.
	
	Задержка реакции -- это	минимальное время от момента, когда входной сигнал изменился, до момента, когда любой из выходов начнет менять свое значение.
	
	\item Что такое критический путь? 
	Почему следует стремиться сократить критические пути в комбинационной части цифровых схем?
	
	Критический путь -- участок схемы с наибольшей задержкой. 
	Он ограничивает скорость, с которой работает микросхема.
	
	\item Опишите предназначение дешифратора и как его реализовать на Verilog.
	Приведите несколько различных вариантов.
	
	Дешифратор выполняет обратную задачу шифратора. 
	Его основное предназначение -- выбор одного из нескольких устройств или сигнала.
	Дешифратор может быть построен на основе операторов case, сдвига, а так же непрерывного присваивания.
	
	\item Как осуществлять оптимизацию при синтезе в САПР Quartus Prime?
	
	Quartus Prime позволяет оптимизировать схему по быстродействию, энергопотреблению, по занимаемой площади кристалла, а так же сбалансированно.
	Опишите различные профили оптимизации в САПР Quartus Prime.
	
	В зависимости от выбранного режима улучшаются одни характеристики, другие же ухудшаются.
	Необходимо оптимизировать тот функционал, который критичен в конкретной ситуации.
	Кроме того, необходимо разработчику не забывать оптимизировать его решение.
	
\end{enumerate}

\section{Выводы по работе}

В ходе работы получен опыт проектирования схем в программе Quartus с помощью языка Verilog.
Полученное устройство было протестировано с помощью бенчтестов в программе Quartus Simulation Waveform editor и ModelSim.
В процессе работы были смоделированы различные шифраторы и дешифраторы, протестированы способы оптимизации схемы, а так же рассчитаны временные параметры схемы с различными способами оптимизации.
В процессе был получен опыт работы с платой DE10-Lite, на которой проверялась работоспособность полученного устройства.

\newpage 
\renewcommand{\refname}{{\normalsize Список использованных источников}} 
\centering 
\begin{thebibliography}{9} 
	\addcontentsline{toc}{section}{\refname} 
	\bibitem{Verilog} Thomas D., Moorby P. The Verilog Hardware Description Language. – Springer Science \& Business Media, 2008.
	\bibitem{citekey} Khor W. Y. et al. Evaluation of FPGA Based QSPI Flash Access Using Partial Reconfiguration //2019 7th International Conference on Smart Computing \& Communications (ICSCC). – IEEE, 2019. – С. 1-5
\end{thebibliography}

\end{document} % конец документа






