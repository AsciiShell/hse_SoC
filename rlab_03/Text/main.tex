%!TEX TS-program = xelatex

% Шаблон документа LaTeX создан в 2018 году
% Алексеем Подчезерцевым
% В качестве исходных использованы шаблоны
% 	Данилом Фёдоровых (danil@fedorovykh.ru) 
%		https://www.writelatex.com/coursera/latex/5.2.2
%	LaTeX-шаблон для русской кандидатской диссертации и её автореферата.
%		https://github.com/AndreyAkinshin/Russian-Phd-LaTeX-Dissertation-Template

\documentclass[a4paper,14pt]{article}

\input{data/preambular.tex}
\begin{document} % конец преамбулы, начало документа
\begin{titlepage}
	\begin{center}
		ФЕДЕРАЛЬНОЕ  ГОСУДАРСТВЕННОЕ АВТОНОМНОЕ \\
		ОБРАЗОВАТЕЛЬНОЕ УЧРЕЖДЕНИЕ ВЫСШЕГО ОБРАЗОВАНИЯ\\
		«НАЦИОНАЛЬНЫЙ ИССЛЕДОВАТЕЛЬСКИЙ УНИВЕРСИТЕТ\\
		«ВЫСШАЯ ШКОЛА ЭКОНОМИКИ»
	\end{center}
	
	\begin{center}
		\textbf{Московский институт электроники и математики}
		
		\textbf{Им. А.Н.Тихонова НИУ ВШЭ}
		
		\textbf{Департамент компьютерной инженерии}
	\end{center}
	\vspace{1ex}	
	\begin{center}
		Подчезерцев Алексей Евгеньевич, группа БИВ174
		
		Солодянкин Андрей Александрович, группа БИВ174
	\end{center}	
	\vspace{1ex}
	\begin{center}
		\textbf{ОТЧЕТ\\
		ПО ЛАБОРАТОРНОЙ РАБОТЕ №1
	}
	\end{center}	
	\vspace{2ex}
	\begin{center}
		по дисциплине «Проектирование систем на кристалле»
	\end{center}

	\vfill
	\begin{center}
		Москва \the\year \, г.
	\end{center}
\end{titlepage}
\tableofcontents
\pagebreak

\section{Задание}

Создать дешифратор адреса в диапазоне 0x873 -- 0x87D, исключая 0x876 и 0x87A.

\section{Выполнение работы}

Составим таблицу истинности для последнего шестнадцатеричного числа в значении адреса (таблица \ref{tab:form}).

\begin{table}[H]
	\caption{Таблица истинности}
	\centering
	\label{tab:form}
	\begin{tabular}{|c|c|c|c?{2pt}c|}
		\hline
		$x_1$ & $x_2$ & $x_3$ & $x_4$ & $y$ \\ \hline
		0     & 0     & 0     & 0     & 0   \\ \hline
		0     & 0     & 0     & 1     & 0   \\ \hline
		0     & 0     & 1     & 0     & 0   \\ \hline
		0     & 0     & 1     & 1     & 1   \\ \hline
		0     & 1     & 0     & 0     & 1   \\ \hline
		0     & 1     & 0     & 1     & 1   \\ \hline
		0     & 1     & 1     & 0     & 0   \\ \hline
		0     & 1     & 1     & 1     & 1   \\ \hline
		1     & 0     & 0     & 0     & 1   \\ \hline
		1     & 0     & 0     & 1     & 1   \\ \hline
		1     & 0     & 1     & 0     & 0   \\ \hline
		1     & 0     & 1     & 1     & 1   \\ \hline
		1     & 1     & 0     & 0     & 1   \\ \hline
		1     & 1     & 0     & 1     & 1   \\ \hline
		1     & 1     & 1     & 0     & 0   \\ \hline
		1     & 1     & 1     & 1     & 0   \\ \hline
	\end{tabular}
\end{table}

Создадим Диаграмму Вейча для заданной функции и оптимизируем заданную функцию (таблица \ref{tab:carno}).

\begin{table}[H]
	\centering
	\caption{Диаграмма Вейча для заданной функции}
	\label{tab:carno}
	\begin{tabular}{cccccc}
		& \multicolumn{2}{c}{$x_1$}                       & \multicolumn{2}{c}{$\bar{x_1}$}                 &                        \\ \cline{2-5}
		\multicolumn{1}{c|}{\multirow{2}{*}{$x_1$}}       & \multicolumn{1}{c|}{1} & \multicolumn{1}{c|}{0} & \multicolumn{1}{c|}{0} & \multicolumn{1}{c|}{1} & $\bar{x_4}$            \\ \cline{2-5}
		\multicolumn{1}{c|}{}                             & \multicolumn{1}{c|}{1} & \multicolumn{1}{c|}{0} & \multicolumn{1}{c|}{1} & \multicolumn{1}{c|}{1} & \multirow{2}{*}{$x_4$} \\ \cline{2-5}
		\multicolumn{1}{c|}{\multirow{2}{*}{$\bar{x_2}$}} & \multicolumn{1}{c|}{1} & \multicolumn{1}{c|}{1} & \multicolumn{1}{c|}{1} & \multicolumn{1}{c|}{0} &                        \\ \cline{2-5}
		\multicolumn{1}{c|}{}                             & \multicolumn{1}{c|}{1} & \multicolumn{1}{c|}{0} & \multicolumn{1}{c|}{0} & \multicolumn{1}{c|}{0} & $\bar{x_4}$            \\ \cline{2-5}
		& $\bar{x_3}$            & \multicolumn{2}{c}{$x_3$}                       & $\bar{x_3}$            &                       
	\end{tabular}
\end{table}

Получим выражение:

\begin{equation}
	y = x_1 \bar{x_3} + x_2 \bar{x_3} + \bar{x_2} x_3 x_4 + \bar{x_1} x_3 x_4
\end{equation}

Создадим по данному выражению схему в редакторе Quartus Prime Lite Edition (рис. \ref{fig:schema}).

\begin{figure}[H]
	\centering
	\includegraphics[width=\linewidth]{image/schema}
	\caption{Итоговая схема}
	\label{fig:schema}
\end{figure}

На рис. \ref{fig:wave} изображена waveform для данного выражения.
Можно убедиться в правильности работы схемы.

\begin{figure}[H]
	\centering
	\includegraphics[width=\linewidth]{image/wave}
	\caption{Результат тестирования}
	\label{fig:wave}
\end{figure}

На рис. \ref{fig:time} изображены временные задержки схемы.

\begin{figure}[H]
	\centering
	\includegraphics[width=\linewidth]{image/time}
	\caption{Задержки схемы}
	\label{fig:time}
\end{figure}

На рис. \ref{fig:demo} изображена ситуация, когда на вход схемы подано значение 7С, а на выходе дешифратора активный сигнал.

\begin{figure}[H]
	\centering
	\includegraphics[width=0.8\linewidth]{image/demo}
	\caption{Результат работы на плате}
	\label{fig:demo}
\end{figure}


\section{Выводы по работе}

В ходе работы были получены выражения в минимизированном варианте, получены результаты моделирования с оптимизацией схемы, рассчитаны временные задержки в с помощью Time Quest Timing Analyzer.
Итоговый вариант был проверен на плате DE10-Lite, наблюдения подтвердили работоспособность устройства и правильность оптимизаций.

\newpage 
\renewcommand{\refname}{{\normalsize Список использованных источников}} 
\centering 
\begin{thebibliography}{9} 
	\addcontentsline{toc}{section}{\refname} 
	\bibitem{Verilog} Thomas D., Moorby P. The Verilog Hardware Description Language. – Springer Science \& Business Media, 2008.
	\bibitem{Quartus} Антонов А., Филиппов А., Золотухо Р. Средства системной отладки САПР Quartus II //Компоненты и технологии. – 2008. – №. 89.
\end{thebibliography}

\end{document} % конец документа