%!TEX TS-program = xelatex

% Шаблон документа LaTeX создан в 2018 году
% Алексеем Подчезерцевым
% В качестве исходных использованы шаблоны
% 	Данилом Фёдоровых (danil@fedorovykh.ru) 
%		https://www.writelatex.com/coursera/latex/5.2.2
%	LaTeX-шаблон для русской кандидатской диссертации и её автореферата.
%		https://github.com/AndreyAkinshin/Russian-Phd-LaTeX-Dissertation-Template

\documentclass[a4paper,14pt]{article}

\input{data/preambular.tex}
\begin{document} % конец преамбулы, начало документа
\begin{titlepage}
	\begin{center}
		ФЕДЕРАЛЬНОЕ  ГОСУДАРСТВЕННОЕ АВТОНОМНОЕ \\
		ОБРАЗОВАТЕЛЬНОЕ УЧРЕЖДЕНИЕ ВЫСШЕГО ОБРАЗОВАНИЯ\\
		«НАЦИОНАЛЬНЫЙ ИССЛЕДОВАТЕЛЬСКИЙ УНИВЕРСИТЕТ\\
		«ВЫСШАЯ ШКОЛА ЭКОНОМИКИ»
	\end{center}
	
	\begin{center}
		\textbf{Московский институт электроники и математики}
		
		\textbf{Им. А.Н.Тихонова НИУ ВШЭ}
		
		\textbf{Департамент компьютерной инженерии}
	\end{center}
	\vspace{1ex}	
	\begin{center}
		Подчезерцев Алексей Евгеньевич, группа БИВ174
		
		Солодянкин Андрей Александрович, группа БИВ174
	\end{center}	
	\vspace{1ex}
	\begin{center}
		\textbf{ОТЧЕТ\\
		ПО ЛАБОРАТОРНОЙ РАБОТЕ №1
	}
	\end{center}	
	\vspace{2ex}
	\begin{center}
		по дисциплине «Проектирование систем на кристалле»
	\end{center}

	\vfill
	\begin{center}
		Москва \the\year \, г.
	\end{center}
\end{titlepage}
\tableofcontents
\pagebreak
\section{Задание}

Основы комбинаторной логики. Маршрут разработки цифровых схем

Исходное выражение:

\begin{equation}
\label{eq:task}
y = \overline{\overline{x_1 + x_2} + x_1 x_3} + \overline{x_1}\overline{x_2}\overline{x_3}\overline{x_4}
\end{equation}

\section{Выполнение работы}

Таблица истинности для заданного выражения: 

\begin{table}[H]
	\begin{center}
		\begin{flushleft}
			\tablecaption{Таблица истинности для выражения \ref{eq:task}}
		\end{flushleft}
	
		\label{tab:logic}
		\begin{tabular}{|c|c|c|c|c|}
			\hline
			$x_1$ & $x_2$ & $x_3$ & $x_4$ & y \\ \hline
			0     & 0     & 0     & 0     & 1 \\ \hline
			1     & 0     & 0     & 0     & 1 \\ \hline
			0     & 1     & 0     & 0     & 1 \\ \hline
			1     & 1     & 0     & 0     & 1 \\ \hline
			0     & 0     & 1     & 0     & 0 \\ \hline
			1     & 0     & 1     & 0     & 0 \\ \hline
			0     & 1     & 1     & 0     & 1 \\ \hline
			1     & 1     & 1     & 0     & 0 \\ \hline
			0     & 0     & 0     & 1     & 0 \\ \hline
			1     & 0     & 0     & 1     & 1 \\ \hline
			0     & 1     & 0     & 1     & 1 \\ \hline
			1     & 1     & 0     & 1     & 1 \\ \hline
			0     & 0     & 1     & 1     & 0 \\ \hline
			1     & 0     & 1     & 1     & 0 \\ \hline
			0     & 1     & 1     & 1     & 1 \\ \hline
			1     & 1     & 1     & 1     & 0 \\ \hline
		\end{tabular}
	\end{center}
\end{table}

На рис. \ref{fig:circuit} изображена реализация выражения \ref{eq:task} в схемотехническом редакторе.

\begin{figure}[H]
	\centering
	\includegraphics[width=0.95\linewidth]{imgs/circuit}
	\caption{Реализация в схемотехническом редакторе}
	\label{fig:circuit}
\end{figure}

Реализация выражения \ref{eq:task} на языке Verilog.

\VerbatimInput{../hdl/lab1.v}

Тестбенч для проверки работоспособности устройства.

\VerbatimInput{../hdl/simulation/testbench.v}

На рис. \ref{fig:modelsim} изображены результаты тестирования устройства в программе ModelSim.

\begin{figure}[H]
	\centering
	\includegraphics[width=0.95\linewidth]{imgs/modelsim}
	\caption{Результат тестирования устройства}
	\label{fig:modelsim}
\end{figure}

\section{Выводы по работе}

В ходе работы получен опыт проектирования схем в программе Quartus с помощью как графического интерфейса, так и языка Verilog.
Полученное устройство было протестировано с помощью бенчтестов в программе ModelSim.
Результаты тестов совпали с таблицей истинности для данного устройства.
В процессе был получен опыт работы с платой DE10-Lite, на которой проверялась работоспособность полученного устройства.

\newpage 
\renewcommand{\refname}{{\normalsize СПИСОК ИСПОЛЬЗОВАННЫХ ИСТОЧНИКОВ}} 
\centering 
\begin{thebibliography}{9} 
	\addcontentsline{toc}{section}{\refname} 
	\bibitem{Verilog} Thomas D., Moorby P. The Verilog Hardware Description Language. – Springer Science \& Business Media, 2008.
	
\end{thebibliography}

\end{document} % конец документа






