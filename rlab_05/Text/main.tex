%!TEX TS-program = xelatex

% Шаблон документа LaTeX создан в 2018 году
% Алексеем Подчезерцевым
% В качестве исходных использованы шаблоны
% 	Данилом Фёдоровых (danil@fedorovykh.ru) 
%		https://www.writelatex.com/coursera/latex/5.2.2
%	LaTeX-шаблон для русской кандидатской диссертации и её автореферата.
%		https://github.com/AndreyAkinshin/Russian-Phd-LaTeX-Dissertation-Template

\documentclass[a4paper,14pt]{article}

\input{data/preambular.tex}
\begin{document} % конец преамбулы, начало документа
\begin{titlepage}
	\begin{center}
		ФЕДЕРАЛЬНОЕ  ГОСУДАРСТВЕННОЕ АВТОНОМНОЕ \\
		ОБРАЗОВАТЕЛЬНОЕ УЧРЕЖДЕНИЕ ВЫСШЕГО ОБРАЗОВАНИЯ\\
		«НАЦИОНАЛЬНЫЙ ИССЛЕДОВАТЕЛЬСКИЙ УНИВЕРСИТЕТ\\
		«ВЫСШАЯ ШКОЛА ЭКОНОМИКИ»
	\end{center}
	
	\begin{center}
		\textbf{Московский институт электроники и математики}
		
		\textbf{Им. А.Н.Тихонова НИУ ВШЭ}
		
		\textbf{Департамент компьютерной инженерии}
	\end{center}
	\vspace{1ex}	
	\begin{center}
		Подчезерцев Алексей Евгеньевич, группа БИВ174
		
		Солодянкин Андрей Александрович, группа БИВ174
	\end{center}	
	\vspace{1ex}
	\begin{center}
		\textbf{ОТЧЕТ\\
		ПО ЛАБОРАТОРНОЙ РАБОТЕ №1
	}
	\end{center}	
	\vspace{2ex}
	\begin{center}
		по дисциплине «Проектирование систем на кристалле»
	\end{center}

	\vfill
	\begin{center}
		Москва \the\year \, г.
	\end{center}
\end{titlepage}
\tableofcontents
\pagebreak

\section{Задание}

Создать конвейерный умножитель.
Изменить схему устройства, добавив собственный блок памяти и арифметический блок.

\section{Выполнение работы}

Создадим проект, в качестве device выберем 5CGXFC9C6F23C7 семейства Cyclone V.
Данное изделие поддерживает инициализацию памяти, хотя, это не обязательно в данной работе.
Создадим необходимые компоненты согласно инструкции, добавим эти модули на схему устройства (рис.~\ref{fig:z1_schema}).

\begin{figure}[H]
	\centering
	\includegraphics[width=\linewidth]{image/z1_schema}
	\caption{Схема параллельного умножителя}
	\label{fig:z1_schema}
\end{figure}

Выполним компиляцию проекта (рис.~\ref{fig:z1_report}).

\begin{figure}[H]
	\centering
	\includegraphics[width=0.5\linewidth]{image/z1_report}
	\caption{Компиляция параллельного умножителя}
	\label{fig:z1_report}
\end{figure}

Построим RTL представление для полученного устройства (рис.~\ref{fig:z1_rtl}).

\begin{figure}[H]
	\centering
	\includegraphics[width=\linewidth]{image/z1_rtl}
	\caption{Общее RTL представление параллельного умножителя}
	\label{fig:z1_rtl}
\end{figure}

Рассмотрим подробнее память (рис.~\ref{fig:z1_rtl2}).
Она состоит из 16 ячеек памяти.

\begin{figure}[H]
	\centering
	\includegraphics[width=\linewidth]{image/z1_rtl2}
	\caption{Подробное RTL представление памяти параллельного умножителя}
	\label{fig:z1_rtl2}
\end{figure}

На рис.~\ref{fig:z1_rtl3} представлена реализация памяти на железе.

\begin{figure}[H]
	\centering
	\includegraphics[width=\linewidth]{image/z1_rtl3}
	\caption{Подробное TMV представление памяти параллельного умножителя}
	\label{fig:z1_rtl3}
\end{figure}

На рис.~\ref{fig:z1_chip_in}~и~\ref{fig:z1_chip_out} изображены входы и выходы сигналов в память.

\begin{figure}[H]
	\centering
	\includegraphics[width=0.5\linewidth]{image/z1_chip_in}
	\caption{Входы в память}
	\label{fig:z1_chip_in}
\end{figure}

\begin{figure}[H]
	\centering
	\includegraphics[width=0.5\linewidth]{image/z1_chip_out}
	\caption{Выходы из памяти}
	\label{fig:z1_chip_out}
\end{figure}

Произведем тестирование параллельного умножителя (рис.~\ref{fig:z1_wave}).
На входы А и Б подаются одинаковые сигналы которые перемножаются как два числа типа integer.
С задержкой 2 тактовых сигнала значения попадают на выход умножителя.
На каждом такте ответ записывается в RAM по некоторому адресу.
В этот же такт происходит считывание на адрес меньше, то есть предыдущего результата.
Итоговая задержка -- 3 такта.
Как можно заметить, на выходе получаем квадраты входных элементов.

\begin{figure}[H]
	\centering
	\includegraphics[width=\linewidth]{image/z1_wave}
	\caption{Тестирование параллельного умножителя}
	\label{fig:z1_wave}
\end{figure}

\subsection{Создание собственного устройства}

Было создано собственное устройство, реализующее функции конвейерного сложения величин.
Память была уменьшена по ширине данных, так как размерность результата суммирования совпадает с размерностью входных данных.
Так же было уменьшено адресное пространство памяти и отключена инициализация за ненадобностью.
Итоговая схема на рис.~\ref{fig:z2_schema}.

\begin{figure}[H]
	\centering
	\includegraphics[width=\linewidth]{image/z2_schema}
	\caption{Схема параллельного сумматора}
	\label{fig:z2_schema}
\end{figure}

Выполним компиляцию проекта (рис.~\ref{fig:z2_report}).

\begin{figure}[H]
	\centering
	\includegraphics[width=0.5\linewidth]{image/z2_report}
	\caption{Компиляция параллельного сумматора}
	\label{fig:z2_report}
\end{figure}

Построим RTL представление для памяти полученного устройства (рис.~\ref{fig:z2_rtl}).

\begin{figure}[H]
	\centering
	\includegraphics[width=\linewidth]{image/z2_rtl}
	\caption{RTL представление памяти параллельного сумматора}
	\label{fig:z2_rtl}
\end{figure}

Построим Technology map view для памяти полученного устройства (рис.~\ref{fig:z2_rtl2}).

\begin{figure}[H]
	\centering
	\includegraphics[width=\linewidth]{image/z2_rtl2}
	\caption{TMV представление памяти параллельного сумматора}
	\label{fig:z2_rtl2}
\end{figure}

Произведем тестирование параллельного сумматора (рис.~\ref{fig:z2_wave}).
На входы А и Б подаются одинаковые сигналы которые складываются как два числа типа integer.
С задержкой 1 тактовый сигнал значения попадают на выход сумматора.
На каждом такте ответ записывается в RAM по некоторому адресу.
В этот же такт происходит считывание на адрес меньше, то есть предыдущего результата.
Итоговая задержка -- 2 такта.
Как можно заметить, на выходе получаем суммы входных элементов.

\begin{figure}[H]
	\centering
	\includegraphics[width=\linewidth]{image/z2_wave}
	\caption{Тестирование параллельного сумматора}
	\label{fig:z2_wave}
\end{figure}

\section{Выводы по работе}

В ходе работы был получен блок параллельного умножителя и блок параллельного сложения.
Оба устройства были протестированы с помощью Quartus waveform, что подтвердило правильность работоспособности устройства.
Так же было выполнено построение RTL и TMV диаграмм для данных устройств.

\newpage 
\renewcommand{\refname}{{\normalsize Список использованных источников}} 
\centering 
\begin{thebibliography}{9} 
	\addcontentsline{toc}{section}{\refname} 
	\bibitem{Verilog} Thomas D., Moorby P. The Verilog Hardware Description Language. – Springer Science \& Business Media, 2008.
	\bibitem{Quartus} Антонов А., Филиппов А., Золотухо Р. Средства системной отладки САПР Quartus II //Компоненты и технологии. – 2008. – №. 89.
\end{thebibliography}

\end{document} % конец документа