%!TEX TS-program = xelatex

% Шаблон документа LaTeX создан в 2018 году
% Алексеем Подчезерцевым
% В качестве исходных использованы шаблоны
% 	Данилом Фёдоровых (danil@fedorovykh.ru) 
%		https://www.writelatex.com/coursera/latex/5.2.2
%	LaTeX-шаблон для русской кандидатской диссертации и её автореферата.
%		https://github.com/AndreyAkinshin/Russian-Phd-LaTeX-Dissertation-Template

\documentclass[a4paper,14pt]{article}

\input{data/preambular.tex}
\begin{document} % конец преамбулы, начало документа
\begin{titlepage}
	\begin{center}
		ФЕДЕРАЛЬНОЕ  ГОСУДАРСТВЕННОЕ АВТОНОМНОЕ \\
		ОБРАЗОВАТЕЛЬНОЕ УЧРЕЖДЕНИЕ ВЫСШЕГО ОБРАЗОВАНИЯ\\
		«НАЦИОНАЛЬНЫЙ ИССЛЕДОВАТЕЛЬСКИЙ УНИВЕРСИТЕТ\\
		«ВЫСШАЯ ШКОЛА ЭКОНОМИКИ»
	\end{center}
	
	\begin{center}
		\textbf{Московский институт электроники и математики}
		
		\textbf{Им. А.Н.Тихонова НИУ ВШЭ}
		
		\textbf{Департамент компьютерной инженерии}
	\end{center}
	\vspace{1ex}	
	\begin{center}
		Подчезерцев Алексей Евгеньевич, группа БИВ174
		
		Солодянкин Андрей Александрович, группа БИВ174
	\end{center}	
	\vspace{1ex}
	\begin{center}
		\textbf{ОТЧЕТ\\
		ПО ЛАБОРАТОРНОЙ РАБОТЕ №1
	}
	\end{center}	
	\vspace{2ex}
	\begin{center}
		по дисциплине «Проектирование систем на кристалле»
	\end{center}

	\vfill
	\begin{center}
		Москва \the\year \, г.
	\end{center}
\end{titlepage}
\tableofcontents
\pagebreak

\section{Задание}

Разработать конвейерный умножитель.
Изучить инструменты Quartus.

\section{Выполнение работы}

Создаем копию проекта работы №5, изменяем настройки device на MAX 10 - 10M50DCF484C7G.
Создадим новую ревизию проекта, настроим Assignment Editor для умножителя (рис.~\ref{fig:setup}).

\begin{figure}[H]
	\centering
	\includegraphics[width=\linewidth]{image/setup}
	\caption{Настройка умножителя}
	\label{fig:setup}
\end{figure}

Выполним компиляцию проекта.
Отчет компиляции изображен на рис.~\ref{fig:compile}.
На нем можно заметить общее число логических элементов, используемых пинов, потребление памяти.
Поэлементные затраты отображены на рис.~\ref{fig:elements}.

\begin{figure}[H]
	\centering
	\includegraphics[width=0.6\linewidth]{image/compile}
	\caption{Результат компиляции проекта}
	\label{fig:compile}
\end{figure}

\begin{figure}[H]
	\centering
	\includegraphics[width=\linewidth]{image/elements}
	\caption{Затраты элементов}
	\label{fig:elements}
\end{figure}

Построим сравнение с предыдущей ревизией (рис.~\ref{fig:diff}).

\begin{figure}[H]
	\centering
	\includegraphics[width=\linewidth]{image/diff}
	\caption{Сравнение ревизий}
	\label{fig:diff}
\end{figure}

Откроем инструмент Pin Planner, назначим входы и выходы для заданной схемы (рис.~\ref{fig:chip_1}).
Зафиксируем положение пинов (рис.~\ref{fig:chip_2}).

\begin{figure}[H]
	\centering
	\includegraphics[width=\linewidth]{image/chip_1}
	\caption{Результат размещения пинов}
	\label{fig:chip_1}
\end{figure}

\begin{figure}[H]
	\centering
	\includegraphics[width=\linewidth]{image/chip_2}
	\caption{Результат фиксации пинов}
	\label{fig:chip_2}
\end{figure}

Вернемся к предыдущей ревизии, импортируем конфигурацию пинов из csv файла, сравним новое положение (красный цвет) с исходным (зеленый цвет) на рис.~\ref{fig:chip_3}.

\begin{figure}[H]
	\centering
	\includegraphics[width=\linewidth]{image/chip_3}
	\caption{Сравнение размещения пинов}
	\label{fig:chip_3}
\end{figure}

Создадим новую ревизию, изменим целевую плату.
Добавим дополнительный пин.
Выполним компиляцию проекта, получим сообщение об ошибке, связанной с банкой (рис.~\ref{error}).

\begin{figure}[H]
	\centering
	\includegraphics[width=\linewidth]{image/error}
	\caption{Ошибка компиляции проекта}
	\label{fig:error}
\end{figure}

\section{Выводы по работе}

В ходе работы был получен блок параллельного умножителя.
Получен опыт работы с различными ревизиями проекта, сравнения информации между версиями.
Также была произведена установка пинов и исследованы особенности работы с банками платы.

\newpage 
\renewcommand{\refname}{{\normalsize Список использованных источников}} 
\centering 
\begin{thebibliography}{9} 
	\addcontentsline{toc}{section}{\refname} 
	\bibitem{Verilog} Thomas D., Moorby P. The Verilog Hardware Description Language. – Springer Science \& Business Media, 2008.
	\bibitem{Quartus} Антонов А., Филиппов А., Золотухо Р. Средства системной отладки САПР Quartus II //Компоненты и технологии. – 2008. – №. 89.
\end{thebibliography}

\end{document} % конец документа