%!TEX TS-program = xelatex

% Шаблон документа LaTeX создан в 2018 году
% Алексеем Подчезерцевым
% В качестве исходных использованы шаблоны
% 	Данилом Фёдоровых (danil@fedorovykh.ru) 
%		https://www.writelatex.com/coursera/latex/5.2.2
%	LaTeX-шаблон для русской кандидатской диссертации и её автореферата.
%		https://github.com/AndreyAkinshin/Russian-Phd-LaTeX-Dissertation-Template

\documentclass[a4paper,14pt]{article}

\input{data/preambular.tex}
\begin{document} % конец преамбулы, начало документа
\begin{titlepage}
	\begin{center}
		ФЕДЕРАЛЬНОЕ  ГОСУДАРСТВЕННОЕ АВТОНОМНОЕ \\
		ОБРАЗОВАТЕЛЬНОЕ УЧРЕЖДЕНИЕ ВЫСШЕГО ОБРАЗОВАНИЯ\\
		«НАЦИОНАЛЬНЫЙ ИССЛЕДОВАТЕЛЬСКИЙ УНИВЕРСИТЕТ\\
		«ВЫСШАЯ ШКОЛА ЭКОНОМИКИ»
	\end{center}
	
	\begin{center}
		\textbf{Московский институт электроники и математики}
		
		\textbf{Им. А.Н.Тихонова НИУ ВШЭ}
		
		\textbf{Департамент компьютерной инженерии}
	\end{center}
	\vspace{1ex}	
	\begin{center}
		Подчезерцев Алексей Евгеньевич, группа БИВ174
		
		Солодянкин Андрей Александрович, группа БИВ174
	\end{center}	
	\vspace{1ex}
	\begin{center}
		\textbf{ОТЧЕТ\\
		ПО ЛАБОРАТОРНОЙ РАБОТЕ №1
	}
	\end{center}	
	\vspace{2ex}
	\begin{center}
		по дисциплине «Проектирование систем на кристалле»
	\end{center}

	\vfill
	\begin{center}
		Москва \the\year \, г.
	\end{center}
\end{titlepage}
\tableofcontents
\pagebreak

\section{Задание}

Составить схему указанного выражения в базисе И, ИЛИ, НЕ. 
Построить временную диаграмму и выполнить моделирование в режимах Functional и Time. 
Оценить аппаратные ресурсы на реализацию схемы и обосновать полученный результат. 
Упростить заданное логическое выражение с помощью алгебры логики. 
Сравнить работу двух схем. 
Запрограммировать учебную плату и продемонстрировать результаты работы на макете (Часть 2).

$$ y = x_1 (\bar{x}_1 + x_2)$$

\section{Выполнение работы}

Таблица истинности.

\begin{table}[H]
	\caption{Таблица истинности}
	\centering
	\begin{tabular}{|c|c|c|}
		\hline
		$x_1$ & $x_2$ & $y$ \\ \hline
		0    & 0    & 0 \\ \hline
		0    & 1    & 0 \\ \hline
		1    & 0    & 0 \\ \hline
		1    & 1    & 1 \\ \hline
	\end{tabular}
\end{table}

Составим упрощенное выражение.

$$ y = x_1 x_2 $$

Создадим блок схему для исходного и упрощенного выражения.

\begin{figure}[H]
	\centering
	\includegraphics[width=\linewidth]{image/schema}
	\caption{Блок схема}
	\label{fig:schema}
\end{figure}

Назначим пины в соответствии с документацией.

\begin{figure}[H]
	\centering
	\includegraphics[width=\linewidth]{image/pins}
	\caption{Назначенные пины}
	\label{fig:pins}
\end{figure}

Проведем симуляцию для данных выражений.

\begin{figure}[H]
	\centering
	\includegraphics[width=0.7\linewidth]{image/waveform}
	\caption{Временные диаграммы}
	\label{fig:waveform}
\end{figure}
 
\section{Вывод}
В ходе работы были смоделированы исходные и упрощенные логические выражения.
Проектирование производилось в логическом базисе И, ИЛИ, НЕ.
В подтверждении эквивалентности были получены одинаковые временные диаграммы для данных функций.
Так же была произведена оценка ресурсов схемы

\newpage 
\renewcommand{\refname}{{\normalsize СПИСОК ИСПОЛЬЗОВАННЫХ ИСТОЧНИКОВ}} 
\centering 
\begin{thebibliography}{9} 
	\addcontentsline{toc}{section}{\refname} 
	\bibitem{Verilog} Thomas D., Moorby P. The Verilog Hardware Description Language. – Springer Science \& Business Media, 2008.
	\bibitem{Quartus} Антонов А., Филиппов А., Золотухо Р. Средства системной отладки САПР Quartus II //Компоненты и технологии. – 2008. – №. 89.
\end{thebibliography}

\end{document} % конец документа