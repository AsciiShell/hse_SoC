%!TEX TS-program = xelatex

% Шаблон документа LaTeX создан в 2018 году
% Алексеем Подчезерцевым
% В качестве исходных использованы шаблоны
% 	Данилом Фёдоровых (danil@fedorovykh.ru) 
%		https://www.writelatex.com/coursera/latex/5.2.2
%	LaTeX-шаблон для русской кандидатской диссертации и её автореферата.
%		https://github.com/AndreyAkinshin/Russian-Phd-LaTeX-Dissertation-Template

\documentclass[a4paper,14pt]{article}

\input{data/preambular.tex}
\begin{document} % конец преамбулы, начало документа
	\begin{titlepage}
	\begin{center}
		ФЕДЕРАЛЬНОЕ  ГОСУДАРСТВЕННОЕ АВТОНОМНОЕ \\
		ОБРАЗОВАТЕЛЬНОЕ УЧРЕЖДЕНИЕ ВЫСШЕГО ОБРАЗОВАНИЯ\\
		«НАЦИОНАЛЬНЫЙ ИССЛЕДОВАТЕЛЬСКИЙ УНИВЕРСИТЕТ\\
		«ВЫСШАЯ ШКОЛА ЭКОНОМИКИ»
	\end{center}
	
	\begin{center}
		\textbf{Московский институт электроники и математики}
		
		\textbf{Им. А.Н.Тихонова НИУ ВШЭ}
		
		\textbf{Департамент компьютерной инженерии}
	\end{center}
	\vspace{1ex}	
	\begin{center}
		Подчезерцев Алексей Евгеньевич, группа БИВ174
		
		Солодянкин Андрей Александрович, группа БИВ174
	\end{center}	
	\vspace{1ex}
	\begin{center}
		\textbf{ОТЧЕТ\\
		ПО ЛАБОРАТОРНОЙ РАБОТЕ №1
	}
	\end{center}	
	\vspace{2ex}
	\begin{center}
		по дисциплине «Проектирование систем на кристалле»
	\end{center}

	\vfill
	\begin{center}
		Москва \the\year \, г.
	\end{center}
\end{titlepage}
	\tableofcontents
	\pagebreak
	\section{Задание}
	
	\begin{enumerate}
		%1
		\item Сравните временную диаграмму симуляционной модели с диаграммой сигналов из документации на микросхему.
		%2
		\item 
		\begin{itemize}
			\item Используя шаги, описанные в Разделе 9.2.5, запустите симуляцию примера 01\_badstyle. Не закрывайте окно симулятора;
			\item Откройте документацию на АЦП ADC081S021 и сравните временную диаграмму в окне симулятора с временной диаграммой из документации;
			\item В среде Quartus Prime откройте пункт меню Tools -> Netlist Viewers -> RTL Viewer;
			\item Найдите экземпляр модуля делителя частоты и сравните его с представленным на Рисунке 9.18;
			\item В среде Quartus Prime откройте пункт меню Tools -> Netlist Viewers -> State Machine	Viewer. Этот же результат можно получить двойным щелчком мыши на желтом прямоугольнике конечного автомата внутри окна RТL Viewer;
			\item Сравните диаграмму конечного автомата с вариантом, разработанным на этапе	проектирования (Рисунок 9.15, Рисунок 9.17);
			\item Откройте файл 01\_badstyle/pmod\_als.v в текстовом редакторе;
			\item Сравните этот код с блок-схемами конечных автоматов (Рисунок 9.1, Рисунок 9.2).
		\end{itemize}
	
	
		%5	
		\item 
		\begin{itemize}
			\item Используя шаги, описание которых дано в Разделе 9.2.5, запустите симуляцию примера 04\_outdelay. Не закрывайте окно симулятора;
			\item В среде Quartus Prime откройте пункт меню Tools -> Netlist Viewers -> RTL Viewer;
			\item Сравните отображаемую схему со схемой из RТL View для примера 0З\_industrial;
			\item Откройте в текстовом редакторе файл 04\_outdelay/pmod\_als.v.
		\end{itemize}

		%6
		\item 
		\begin{itemize}
			\item Иcпользуя шаги, описанные в Разделе 9.2.5, запустите симуляцию примера 05\_optimized. Не закрывайте окно симулятора;
			\item В среде Quartus Prime откройте пункт меню Tools -> Netlist Viewers -> RTL Viewer;
			\item Сравните отображаемую схему со схемой из RТL View для примера 03\_industrial;
			\item Отдельно сравните модули делителей частоты;
			\item Откройте в текстовом редакторе файл 05\_optimized/pmod\_als.v.
		\end{itemize}
		
		%7
		\item Сравните результаты синтеза делителей частоты при различных подходах к проектированию ( Рисунок 9.18, Рисунок 9.26);
		
		
		%8	
		\item 
		\begin{itemize}
			\item Используя шаги, описание которых приведено в Разделе 9.2.5, запустите симуляцию примера Oб\_counter. Не закрывайте окно симулятора;
			\item В среде Quartus Prime откройте пункт меню Tools -> Netlist Viewers -> RTL Viewer;
			\item Сравните отображаемую схему со схемой из RТL View для примера 03\_industrial;
			\item Откройте в текстовом редакторе файл 06\_counter/pmod\_als.v.
		\end{itemize}
	
		%9
		\item Проанализируйте Листинг 9.12, отследите сигналы sck, cs, sampleBit и valueDone в окне симулятора и на временной диаграмме ( Рисунок 9.28). Особое внимание уделите тому, как они формируются. Обоснуйте, как оптимизация автомата сказалась на затратах ресурсов чипа и проиллюстрируйте это отчетами компилятора после сборки проекта.
	\end{enumerate}
	
	\section{Дополнительные задания}
	
	\subsection{Задание 1}
	
	Сравните временную диаграмму симуляционной модели с	диаграммой сигналов из документации на микросхему. 		
	
	На рис. \ref{fig:9.7} приведена временная диаграмма из документации к микросхеме ADC081S021, на рис. \ref{fig:9.11} диаграмма симуляционной модели микросхемы.
	
	\begin{figure}[H]
		\centering
		\includegraphics[width=0.9\linewidth]{images/9_7}
		\caption{ADC081S021 Временная диаграмма обмена данными из документации}
		\label{fig:9.7}
	\end{figure}
	
	
	\begin{figure}[H]
		\centering
		\includegraphics[width=0.9\linewidth]{images/9_11}
		\caption{Временная диаграмма симуляционной модели микросхемы ADC081S021}
		\label{fig:9.11}
	\end{figure}

	На обеих временных диаграммах изображены 3 временные характеристики: $cs$ - chip select, сигнал выбора микросхемы, $sclk(sck)$ - тактовый импульс, $sdata(sdo)$ - slave data out, данные, получаемые из микросхемы.
	
	$sclk(sck)$ мало чем может отличаться, т.к. это тактовый импульс.
	
	$cs$ на обоих графиках установлен в значение логического нуля на протяжении 16 импульсов.
	
	$sdata(sdo)$ на обоих графиках меняет свое состояние по заднему фронту тактового импульса. При этом после первых трех тактовых импульсов наблюдаются нулевые биты, далее идут 8 информационных битов, далее идут 4 нулевых бита и после 16 тактового импульса сигнал принимает высокоимпедансное состояние.
	
	Разница в графиках только в длительности фронтов.	
		
	\subsection{Задание 2}

	После запуска симуляции в modelSim была получена диаграмма, представленная на рис. \ref{fig:z15msimwvf}.
	
	\begin{figure}[H]
		\centering
		\includegraphics[width=0.9\linewidth]{images/z1_5_msim_wvf}
		\caption{Временная диаграмма для 01\_badstyle}
		\label{fig:z15msimwvf}
	\end{figure}

	При сравнении диаграмм (рис. \ref{fig:9.7} и рис. \ref{fig:z15msimwvf}) видно, что микросхема ADC081S021 реализована верно. Их сравнение аналогично сравнению в задании 1.
	
	На рис. \ref{fig:z15rtl} RTL представление менеджера сессий для 01\_badstyle.
	
	\begin{figure}[H]
		\centering
		\includegraphics[width=0.9\linewidth]{images/z1_5_rtl}
		\caption{RTL представление менеджера сессий для 01\_badstyle}
		\label{fig:z15rtl}
	\end{figure}

	На рис. \ref{fig:z15rtlclkdivider} RTL представление делителя частоты для 01\_badstyle.
	
	\begin{figure}[H]
		\centering
		\includegraphics[width=0.9\linewidth]{images/z1_5_rtl_clk_divider}
		\caption{RTL представление делителя частоты для 01\_badstyle}
		\label{fig:z15rtlclkdivider}
	\end{figure}

	На рис. \ref{fig:z15auto} диаграмма состояний менеджера сессий для 01\_badstyle.
	
	\begin{figure}[H]
		\centering
		\includegraphics[width=0.6\linewidth]{images/z1_5_auto}
		\caption{Диаграмма состояний менеджера сессий для 01\_badstyle}
		\label{fig:z15auto}
	\end{figure}
	
	На рис. \ref{fig:z15autoclkdivider} диаграмма состояний делителя частоты для 01\_badstyle.
	
	\begin{figure}[H]
		\centering
		\includegraphics[width=0.6\linewidth]{images/z1_5_auto_clk_divider}
		\caption{Диаграмма состояний делителя частоты для 01\_badstyle}
		\label{fig:z15autoclkdivider}
	\end{figure}

	Все диаграммы совпадают с диаграммами представленными в методичке.
	
	Данный фрагмент кода описывает схему переходов для менеджера сессий.
	
	{\small \VerbatimInput{verilog/z1_5_men_sess.v}}

	Данный фрагмент кода описывает схему переходов для делителя частоты.
	
	{\small \VerbatimInput{verilog/z1_5_clk_divider.v}}
	
	\subsection{Задание 3}
	
	Проверить компромиссный стиль кодирования.
	
	Была выполнена симуляция схемы (рис.~\ref{fig:9_3_wave}), далее код был скомпилирован (рис.~\ref{fig:9_3_compilation}).
	Было выполнено RTL моделирование (рис.~\ref{fig:9_3_rtl}~и~\ref{fig:9_3_rtl2}), а так же построена диаграмма конечного автомата (рис.~\ref{fig:9_3_state}).
	RTL представление усложнилось, по сравнению с рис.~\ref{fig:z15rtl}.
	
	Диаграмма конечного автомата совпадает с примером диаграммы состояний менеджера сессий.
	При анализе исходного кода можно выделить блок, отвечающий за логику переходов, за вычисление следующего состояния, вычисления следующего состояния.
	
	\begin{figure}[H]
		\centering
		\includegraphics[width=0.9\linewidth]{images/9_3_wave}
		\caption{Результат симуляции компромиссного стиля кодирования}
		\label{fig:9_3_wave}
	\end{figure}
	
	\begin{figure}[H]
		\centering
		\includegraphics[width=0.5\linewidth]{images/9_3_compilation}
		\caption{Результат компиляции компромиссного стиля кодирования}
		\label{fig:9_3_compilation}
	\end{figure}
	
	\begin{figure}[H]
		\centering
		\includegraphics[width=0.9\linewidth]{images/9_3_rtl}
		\caption{RTL представление компромиссного стиля кодирования}
		\label{fig:9_3_rtl}
	\end{figure}
	
	\begin{figure}[H]
		\centering
		\includegraphics[width=0.9\linewidth]{images/9_3_rtl2}
		\caption{Подробное RTL представление компромиссного стиля кодирования}
		\label{fig:9_3_rtl2}
	\end{figure}
	
	\begin{figure}[H]
		\centering
		\includegraphics[width=0.9\linewidth]{images/9_3_state}
		\caption{State machine view компромиссного стиля кодирования}
		\label{fig:9_3_state}
	\end{figure}

	\subsection{Задание 4}

	Проверить стиль на примитивах.
	
	Была выполнена симуляция схемы (рис.~\ref{fig:9_4_wave}), далее код был скомпилирован (рис.~\ref{fig:9_4_compilation}).
	Было выполнено RTL моделирование (рис.~\ref{fig:9_4_rtl}~и~\ref{fig:9_4_rtl2}), конечный автомат в данном примере не распознался.
	Общий вид RTL не отличается от рис.~\ref{fig:9_3_rtl}, однако при подробном анализе можно заметить, что в данном варианте схема состоит из множества малых частей, а так же стала больше при полном отображении всех компонентов.
	Кроме того, в данном варианте компилятор не распознает конечный автомат, он не отображается ни в RTL, ни в State machine view.
	
	Особенность данной схемы в том, что каждый отдельный элемент является отдельным блоком.
	Это приводит к упрощению схемы для разработчика, а так же снижает затраты на отладку и тестирование компонентов. 
	
	\begin{figure}[H]
		\centering
		\includegraphics[width=0.9\linewidth]{images/9_4_wave}
		\caption{Результат симуляции стиля на примитивах}
		\label{fig:9_4_wave}
	\end{figure}
	
	\begin{figure}[H]
		\centering
		\includegraphics[width=0.5\linewidth]{images/9_4_compilation}
		\caption{Результат компиляции стиля на примитивах}
		\label{fig:9_4_compilation}
	\end{figure}
	
	\begin{figure}[H]
		\centering
		\includegraphics[width=0.9\linewidth]{images/9_4_rtl}
		\caption{RTL представление стиля на примитивах}
		\label{fig:9_4_rtl}
	\end{figure}
	
	\begin{figure}[H]
		\centering
		\includegraphics[width=0.9\linewidth]{images/9_4_rtl2}
		\caption{Подробное RTL представление стиля на примитивах}
		\label{fig:9_4_rtl2}
	\end{figure}

	\subsection{Задание 5}
	
	После запуска симуляции в modelSim была получена диаграмма, представленная на рис. \ref{fig:z5msimwvf}.
	
	\begin{figure}[H]
		\centering
		\includegraphics[width=0.9\linewidth]{images/z5_msim_wvf}
		\caption{Временная диаграмма для 04\_outdelay}
		\label{fig:z5msimwvf}
	\end{figure}
	
	На рис. \ref{fig:z5rtl} RTL представление менеджера сессий для 04\_outdelay.
	
	\begin{figure}[H]
		\centering
		\includegraphics[width=0.9\linewidth]{images/z5_rtl}
		\caption{RTL представление менеджера сессий для 04\_outdelay}
		\label{fig:z5rtl}
	\end{figure}

	\begin{figure}[H]
		\centering
		\includegraphics[width=0.9\linewidth]{images/9_5}
		\caption{RTL представление менеджера сессий для 0З\_industrial}
		\label{fig:95}
	\end{figure}
	
	RTL представления 04\_outdelay (рис. \ref{fig:z5rtl}) и  0З\_industrial (рис. \ref{fig:95}) похожи, но 04\_outdelay имеет на выходе дополнительный регистр.
	
	\subsection{Задание 6}
	
	После запуска симуляции в modelSim была получена диаграмма, представленная на рис. \ref{fig:z6msimwvf}.
	
	\begin{figure}[H]
		\centering
		\includegraphics[width=0.9\linewidth]{images/z6_msim_wvf}
		\caption{Временная диаграмма для 05\_optimized}
		\label{fig:z6msimwvf}
	\end{figure}
	
	На рис. \ref{fig:z6rtl} представлено RTL представление менеджера сессий для 05\_optimized.
	
	\begin{figure}[H]
		\centering
		\includegraphics[width=0.9\linewidth]{images/z6_rtl}
		\caption{RTL представление менеджера сессий для 05\_optimized}
		\label{fig:z6rtl}
	\end{figure}
	
	RTL представления 05\_optimized (рис. \ref{fig:z6rtl}) и  0З\_industrial (рис. \ref{fig:95}) по-прежнему отличаются одним дополнительным регистром.
	Кроме этого появилось еще одно различие, делитель частоты для 05\_optimized (рис. \ref{fig:z6trlclkdiv}) стал проще чем делитель частоты для 0З\_industrial (рис. \ref{fig:z15rtlclkdivider1}).
	
	\begin{figure}[H]
		\centering
		\includegraphics[width=0.7\linewidth]{images/z6_trl_clk_div}
		\caption{RTL представление делителя частоты для 05\_optimized}
		\label{fig:z6trlclkdiv}
	\end{figure}

	\begin{figure}[H]
		\centering
		\includegraphics[width=0.7\linewidth]{images/z1_5_rtl_clk_divider}
		\caption{RTL представление делителя частоты для 0З\_industrial}
		\label{fig:z15rtlclkdivider1}
	\end{figure}
	
	\subsection{Задание 7}
	
	\begin{figure}[H]
		\begin{minipage}[h]{0.4\linewidth}
			\center{\includegraphics[width=1\linewidth]{z1_5_compilation}} 01\_badstyle \\
		\end{minipage}
		\hfill
		\begin{minipage}[h]{0.4\linewidth}
			\center{\includegraphics[width=1\linewidth]{9_3_compilation}} \\02\_compromise
		\end{minipage}
		\vfill
		\begin{minipage}[h]{0.4\linewidth}
			\center{\includegraphics[width=1\linewidth]{9_4_compilation}} 03\_industrial \\
		\end{minipage}
		\hfill
		\begin{minipage}[h]{0.4\linewidth}
			\center{\includegraphics[width=1\linewidth]{z5_compilation}} 04\_outdelay \\
		\end{minipage}
		\vfill
		\begin{minipage}[h]{0.4\linewidth}
			\center{\includegraphics[width=1\linewidth]{z6__compilation}} 05\_optimized \\
		\end{minipage}
		\hfill
		\begin{minipage}[h]{0.4\linewidth}
			\center{\includegraphics[width=1\linewidth]{z8__compilation}} 06\_counter \\
		\end{minipage}
		\caption{Результаты компиляций всех реализаций}
		\label{fig:compilations}
	\end{figure}
	
	На рис. \ref{fig:compilations} представлены результаты компиляции всех реализаций. Самым экономичным является  последняя реализация с двоичным счетчиком (40 регистров и 60 логических элементов).
	
	Самой неэкономичной реализацией оказалась первая версия (53 регистра и 94 логических элемента).  
	
	
	\subsection{Задание 8}
	
	После запуска симуляции в modelSim была получена диаграмма, представленная на рис. \ref{fig:z8msimwvf}.
	
	\begin{figure}[H]
		\centering
		\includegraphics[width=0.9\linewidth]{images/z9}
		\caption{Временная диаграмма для 06\_counter}
		\label{fig:z8msimwvf}
	\end{figure}
	
	На рис. \ref{fig:z8rtl} RTL представление менеджера сессий для 06\_counter.
	
	\begin{figure}[H]
		\centering
		\includegraphics[width=0.9\linewidth]{images/z8_rtl}
		\caption{RTL представление менеджера сессий для 06\_counter}
		\label{fig:z8rtl}
	\end{figure}
		
	RTL представления 06\_counter (рис. \ref{fig:z8rtl}) стало в разы проще, т.к. подход к проектированию конечного автомата на основе счетчика	применяется ко всему модулю взаимодействия с датчиком освещенности.
	
	\subsection{Задание 9}
	
	Отследим отследите сигналы sck, cs, sampleBit и valueDone на рис. \ref{fig:z8msimwvf}.
	
	\begin{enumerate}
		
		\item sck - тактовый сигнал с пониженной частотой, объявлено как 3 бит в счетчике: $assign sck = cnt [3]$;
		
		\item cs - chip select, сигнал выбора микросхемы, инвертированный;
		
		\item sampleBit - старый тактовый сигал с счетчиком;
		
		\item valueDone - сигнал окончания передачи информационной части.
	\end{enumerate}

	По поводу ресурсозатратности было в предыдущем задании.
	
	\section{Задания для самостоятельной работы}

	\subsection{Задание 1}

	\section{Контрольные вопросы}
	
	\begin{enumerate}
		
		\item   
	\end{enumerate}
	
	\section{Выводы по работе}
	
	В ходе работы получен опыт проектирования схем в программе Quartus с помощью языка Verilog.
	Полученное устройство было протестировано с помощью бенчтестов в программе Quartus Simulation Waveform editor.
	В процессе работы были смоделированы конечные автоматы Мура и Мили.
	%В процессе работы были смоделированы различные шифраторы и дешифраторы, протестированы способы оптимизации схемы, а так же рассчитаны временные параметры схемы с различными способами оптимизации.
	В процессе был получен опыт работы с платой DE10-Lite, на которой проверялась работоспособность полученного устройства.
	
	\newpage 
	\renewcommand{\refname}{{\normalsize Список использованных источников}} 
	\centering 
	\begin{thebibliography}{9} 
		\addcontentsline{toc}{section}{\refname} 
		\bibitem{Verilog} Thomas D., Moorby P. The Verilog Hardware Description Language. – Springer Science \& Business Media, 2008.
		\bibitem{citekey} Khor W. Y. et al. Evaluation of FPGA Based QSPI Flash Access Using Partial Reconfiguration //2019 7th International Conference on Smart Computing \& Communications (ICSCC). – IEEE, 2019. – С. 1-5
	\end{thebibliography}
	
\end{document} % конец документа
